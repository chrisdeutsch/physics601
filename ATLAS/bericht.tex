% PAKETE UND DOKUMENTKONFIGURATION
\documentclass[11pt, a4paper]{article}

% Encoding für Umlaute
\usepackage[utf8]{inputenc}
\usepackage[T1]{fontenc}

% Silbentrennung
\usepackage[ngerman]{babel}

% erweiterte Matheumgebungen und Formelnummer mit Sectionnummer
\usepackage{amsmath}
\numberwithin{equation}{section}

% Braket Notation
\usepackage{braket}
\usepackage{isotope}
\usepackage[version=3]{mhchem}
\usepackage{tensor}
\usepackage{slashed}

% zusätzliche mathematische Schriftarten
\usepackage{amsfonts}

% verschiedene mathematische Symbole
\usepackage{amssymb}

% Einheiten setzen z.B. \SI{10}{\kilo\gram\meter\per\second\squared}
% Fehler: \SI{10 +- 0,2e-4}{\metre}
\usepackage{siunitx}
\sisetup{
  output-decimal-marker={,},
  separate-uncertainty
}

% Einheitendefinitionen
\DeclareSIUnit{\skt}{Skt.}
\DeclareSIUnit{\gauss}{G}
\DeclareSIUnit{\division}{div.}

% Operatordefinitionen
\DeclareMathOperator{\erf}{erf}

% Randbreiten
\usepackage[left=3.5cm,right=3.5cm,top=3cm,bottom=3cm,twoside]{geometry}

% Bilder einfügen
\usepackage{graphicx}

% Textfarbe
\usepackage{color}

% Verweise innerhalb des Dokuments
\usepackage{hyperref}
\hypersetup{
	colorlinks = true,
	allcolors = {black}
}

% bessere Tabellenlayouts
\usepackage{booktabs}
\usepackage{multirow}
\usepackage{multicol}

% Seitenlayout (Kopfzeile)
\usepackage{fancyhdr}

% Float Barriers
\usepackage{placeins}

% Pakete für gedrehte Subfigures
\usepackage{caption}
\usepackage{subcaption}
\usepackage{rotating}

% Paket für textumflossene Abbildungen und Tabellen
\usepackage{wrapfig}

\usepackage{float}

% Caption-Setup
\captionsetup{font={small}}
\renewcommand{\thefigure}{\thesection.\arabic{figure}}
\renewcommand{\thesubfigure}{\alph{subfigure}}
\renewcommand{\thetable}{\thesection.\arabic{table}}
\renewcommand{\thesubtable}{\alph{subtable}}

% Manuelle Silbentrennung
\hyphenation{Sig-nal-ver-hal-ten Szin-til-la-tions-de-tek-tors}

% Tiefe des Inhaltsverzeichnisses (Level: 1 sections, 2 subsections,
% 3 subsubsections)
\setcounter{tocdepth}{3}

% FANCYHDR SETUP
\pagestyle{fancy}
\fancyhead[EL,OR]{\thepage}
\fancyhead[ER]{\leftmark}
\fancyhead[OL]{\rightmark}
\setlength{\headheight}{13.6pt}

\renewcommand{\sectionmark}[1]{
\markboth{\thesection{} #1}{\thesection{} #1}
}
\renewcommand{\subsectionmark}[1]{
\markright{\thesubsection{} #1}
}

\newcommand{\korr}[1]{{\color{red}(#1)}}

% DOKUMENTINFORMATIONEN
\title{E214 \\ ATLAS}

\author{Christopher Deutsch\footnote{christopher.deutsch@uni-bonn.de} \and Christian Bespin\footnote{christian.bespin@uni-bonn.de}}

\date{\today}

\begin{document}

\begin{titlepage}

\maketitle

% DURCHFÜHRUNGSDATUM UND TUTOR
\begin{center}
\begin{tabular}{l r}
Durchführung: & 7./8. März 2016 \\
Gruppe: & P8 \\
Tutor: & Elisabeth Schopf
\end{tabular}
\end{center}

% ZUSAMMENFASSUNG
\begin{abstract}
\noindent Zusammenfassung hier
\end{abstract}

\end{titlepage}

% INHALTSVERZEICHNIS
\tableofcontents
% Neue Seite nach TOC
\newpage

% INHALT VERSUCHSPROTOKOLL
\section{Theorie}

\subsection{Einführung}

\subsubsection{Standardmodell}

Das Standardmodell beschreibt die Elementarteilchen und die Wechselwirkungen zwischen ihnen, mit Ausnahme der Gravitation und vereint somit die Erkenntnisse der Teilchenphysik.
Es gilt seit der Entdeckung des Higgs-Bosons als abgeschlossen.
Im Rahmen des Standardmodells können die einzelnen Elementarteilchen in Gruppen zu Leptonen, Quarks und Kraftteilchen zusammengefasst werden.
Leptonen und Quarks sind Fermionen, d.h. sie sind Spin~-$1/2$-Teilchen und unterscheiden sich durch ihre Ladung.
Während Leptonen, deren bekanntester Vertreter das Elektron ist, eine Ladungszahl~$1$ haben und die zugehörigen Neutrinos ungeladen sind, tragen Quarks drittelzahlige Ladungen~($1/3$ bzw.~$2/3$).
Durch Eichbosonen, wie Photon, Gluon, Z- und W-Bosonen wirken Kräfte zwischen den Quarks und Leptonen.
Das Higgs-Boson fällt aus dieser Einteilung heraus, da es eine Folge der spontanen Symmetriebrechung ist, die mit der Ursache von Masse assoziiert wird.

Durch diese Einteilung lässt sich eine Übersicht der einzelnen Elementarteilchen des Standardmodells gut wie in Abbildung \korr{REF} grafisch darstellen.

\subsubsection{LHC und ATLAS}

Der Large Hadron Collider (LHC) der europäischen Organisation für Kernforschung (CERN) ist der zur Zeit weltgrößte Teilchenbeschleuniger mit einer Schwerpunktsenergie von bis zu~\SI{14}{TeV}.
Dadurch können physikalische \korr{hethetheth mir fehlt ein Wort} bei hohen Energien überprüft und möglicherweise neue Beobachtungen gemacht werden.
Hierzu kommen vier Detektoren zum Einsatz, von denen zwei der Untersuchung hochenergetischer Ereignisse auf der Tera-Skala dienen.
Einer dieser Detektoren ist der ATLAS-Detektor, mit dem die in diesem Versuch analysierten Daten aufgenommen wurden.
\korr{Hier Beschreibung und Aufbau des Detektors?}

\subsection{Kinematik}

\subsubsection{Lorentzvektoren}
Ein Lorentzvektor~$\mathbf{x}$ ist ein vierdimensionaler Vektor mit reellen Komponenten, welche bezüglich einer Lorentztransformation~$\Lambda$ gemäß
\begin{align*}
	{x^\prime}^\mu = \tensor{\Lambda}{^\mu_\nu} x^\nu \qquad {x^\prime}_\mu = \tensor{\left(\Lambda^{-1}\right)}{^\nu_\mu} x_\nu
\end{align*}
transformieren.
Dadurch ist das Minkowski-Produkt zweier Lorentzvektoren
\begin{align*}
	\mathbf{x} \cdot \mathbf{y} = \tensor{g}{_\mu_\nu} x^\mu y^\nu \quad \text{mit} \quad g = \mathrm{diag}(1,-1,-1,-1)
\end{align*}
invariant unter Lorentztransformationen.

Ein solcher Lorentzvektor ist der sog.\ Viererimpuls~$\mathbf{p} = \left(E, \vec{p}\right)$, bestehend aus der Gesamtenergie~$E$ und dem Impuls~$\vec{p}$ im jeweiligen Bezugssystem.
Dieser ist essenziell für kinematische Betrachtungen in der Teilchenphysik und tritt häufig in Form von lorentzinvarianten Skalarprodukten auf, die in einem beliebigen Bezugssystem (z.B.\ im Schwerpunktssystem) ausgewertet werden können.

Beispiele dafür sind die invariante Masse eines Teilchens~$m_\mathrm{inv}^2 = \mathbf{p}^2$ oder die Mandelstam-Variable~$s = (\mathbf{p_1} + \mathbf{p_2})^2$, welche der quadratischen Schwerpunktsenergie zweier Teilchen entspricht.

\subsubsection{Rapidität und Pseudorapidität}
Die Rapidität entlang der Strahlachse~$z$ eines Teilchenbeschleunigers ist definiert als
\begin{align*}
	y = \frac{1}{2}\ln\left( \frac{E+p_z}{E-p_z} \right)
\end{align*}
mit der Energie~$E$ und dem Impuls~$p_z$ entlang der Strahlachse.
Wie das Lorentz-Beta~$\beta$ ist die Rapidität ein Maß für die Geschwindigkeit eines Teilchens.
Jedoch ist die Rapidität im Vergleich zu $\beta$ gegenüber Lorentz-Boosts entlang der Strahlachse lorentzinvariant und somit ein additives Geschwindigkeitsmaß, wie es bereits aus der klassischen Mechanik bekannt ist.

Für Teilchen mit verschwindender Masse ($E \approx p$) geht die Rapidität~$y$ in die Pseudorapidität~$\eta$ über
\begin{align*}
	\eta = \frac{1}{2} \ln\left( \frac{1 + \cos\theta}{1 - \cos\theta} \right) = - \ln\left( \tan\frac{\theta}{2} \right)\text{,}
\end{align*}
welche nur noch eine Funktion des Polarwinkels $\theta$ im jeweiligen Bezugssystem darstellt.
Im Gegensatz zu dem Polarwinkel~$\theta$ sind Pseudorapiditätsdifferenzen lorentzinvariant und somit ein bevorzugtes Maß für die Beschreibung von Ereignissen in Teilchendetektoren.
Beispielsweise eine Pseudorapidität von $\eta = \num{0}$ einem Polarwinkel von $\theta = \SI{90}{\degree}$ (senkrecht zur Strahlachse) und positive/negative Pseudorapiditäten entsprechen Winkeln in/entgegen der Richtung der Strahlachse.

\subsubsection{Hadronen-Kollision}
Bei Kollisionen an Hadronenbeschleunigern wie dem LHC gilt es zu beachten, dass die eigentliche Kollision nicht zwischen den Hadronen, sondern dessen Partonen stattfindet.
Im Partonmodell besteht jedes Hadron neben Valenzquarks auch aus Gluonen und paarerzeugten Quark-Antiquark-Paaren.
Jedes Parton trägt dabei einen Impulsbruchteil, welcher im Bezugssystem in dem das Hadron einen unendlichen longitudinalen Impuls hat (d.h.\ transversale Impulse können vernachlässigt werden) durch die Bjorken'sche Skalenvariable~$x$ gegeben ist.

Zur Charakterisierung von Hadronen-Kollisionen ist es demnach erforderlich die Wahrscheinlichkeitsverteilung, dass ein Parton den Impulsbruchteil~$x$ des Hadrons trägt, zu bestimmen.
Diese Verteilungen, welche man Partonverteilungsfunktionen~$f(x)$ nennt, wurden für die verschiedenen Partonentypen~($u, \bar{u}, d, \dots, g$) an mehreren Teilchenbeschleunigern vermessen.

Dennoch sind die spezifischen Impulsbruchteile~$x_1, x_2$ der kollidierenden Partonen in der Regel unbekannt, so dass der Transversalimpuls~$p_\mathrm{T}$, welcher beispielsweise im durch den Krümmungsradius geladener Teilchen im äußeren Magnetfeld bestimmt werden kann, für die Auswertung herangezogen wird.
Darüber hinaus kann aus der Summe der Transversalimpulse aller Teilchen im Endzustand, welche gemäß der Impulserhaltung für frontal zusammenstoßende Partonen zu null summieren müssen, fehlender Transversalimpuls reproduziert werden, der auf nicht-detektierbare Teilchen wie Neutrinos zurückzuführen ist.

Da der Transversalimpuls für ungeladene Teilchen wie beispielsweise Photonen nicht durch den Krümmungsradius im Magnetfeld vermessen werden kann, wird stattdessen auf die Energieeinträge in den Kalorimetern zurückgegriffen und die Größe der fehlenden transversalen Energie~$\slashed{E}_\mathrm{T}$
\begin{align*}
	\slashed{E}_\mathrm{T} = - \sum_{i} E^i \sin\theta_i \, \vec{n}_{i,\perp}
\end{align*}
mit den Energieeinträgen~$E_i$ des Kalorimeters unter Polarwinkeln~$\theta_i$  und azimuthalen Einheitsvektoren~$\vec{n}_{i,\perp}$ auf den jeweiligen Eintrag, definiert.



\subsection{Neue Physik}

\subsubsection{Higgs-Boson}

\subsubsection{Supersymmetrie}


\section{Beschreibung der Software (ist sowas wie Versuchsaufbau)}


\FloatBarrier
% BIBLIOGRAPHIE
\vspace{\fill}
% Maximale Anzahl der Einträge in Klammer
% Zitieren mit \cite{lamport94}
\begin{thebibliography}{19}
\bibitem{leo}
	W. R. Leo,
	\emph{Techniques for Nuclear and Particle Physics Experiments},
	Springer 1994

\end{thebibliography}

% APPENDIX
\begin{appendix}
\newpage
\section{Anhang}

\FloatBarrier
\subsection{Section in Appendix}
Appendix
\FloatBarrier

\end{appendix}

\end{document}
