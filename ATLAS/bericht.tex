% PAKETE UND DOKUMENTKONFIGURATION
\documentclass[11pt, a4paper]{article}

% Encoding für Umlaute
\usepackage[utf8]{inputenc}
\usepackage[T1]{fontenc}

% Silbentrennung
\usepackage[ngerman]{babel}

% erweiterte Matheumgebungen und Formelnummer mit Sectionnummer
\usepackage{amsmath}
\numberwithin{equation}{section}

% Braket Notation
\usepackage{braket}
\usepackage{isotope}
\usepackage[version=3]{mhchem}
\usepackage{tensor}
\usepackage{slashed}

% zusätzliche mathematische Schriftarten
\usepackage{amsfonts}

% verschiedene mathematische Symbole
\usepackage{amssymb}

% Einheiten setzen z.B. \SI{10}{\kilo\gram\meter\per\second\squared}
% Fehler: \SI{10 +- 0,2e-4}{\metre}
\usepackage{siunitx}
\sisetup{
  output-decimal-marker={,},
  separate-uncertainty
}

% Einheitendefinitionen
\DeclareSIUnit{\skt}{Skt.}
\DeclareSIUnit{\gauss}{G}
\DeclareSIUnit{\division}{div.}

% Operatordefinitionen
\DeclareMathOperator{\erf}{erf}

% Randbreiten
\usepackage[left=3.5cm,right=3.5cm,top=3cm,bottom=3cm,twoside]{geometry}

% Bilder einfügen
\usepackage{graphicx}

% Textfarbe
\usepackage{color}

% Verweise innerhalb des Dokuments
\usepackage{hyperref}
\hypersetup{
	colorlinks = true,
	allcolors = {black}
}

% bessere Tabellenlayouts
\usepackage{booktabs}
\usepackage{multirow}
\usepackage{multicol}

% Seitenlayout (Kopfzeile)
\usepackage{fancyhdr}

% Float Barriers
\usepackage{placeins}

% Pakete für gedrehte Subfigures
\usepackage{caption}
\usepackage{subcaption}
\usepackage{rotating}

% Paket für textumflossene Abbildungen und Tabellen
\usepackage{wrapfig}

\usepackage{float}

% Caption-Setup
\captionsetup{font={small}}
\renewcommand{\thefigure}{\thesection.\arabic{figure}}
\renewcommand{\thesubfigure}{\alph{subfigure}}
\renewcommand{\thetable}{\thesection.\arabic{table}}
\renewcommand{\thesubtable}{\alph{subtable}}

% Manuelle Silbentrennung
\hyphenation{Par-ton-ver-teil-ungs-funk-tio-nen}

% Tiefe des Inhaltsverzeichnisses (Level: 1 sections, 2 subsections,
% 3 subsubsections)
\setcounter{tocdepth}{3}

% FANCYHDR SETUP
\pagestyle{fancy}
\fancyhead[EL,OR]{\thepage}
\fancyhead[ER]{\leftmark}
\fancyhead[OL]{\rightmark}
\setlength{\headheight}{13.6pt}

\renewcommand{\sectionmark}[1]{
\markboth{\thesection{} #1}{\thesection{} #1}
}
\renewcommand{\subsectionmark}[1]{
\markright{\thesubsection{} #1}
}

\newcommand{\korr}[1]{{\color{red}(#1)}}

% DOKUMENTINFORMATIONEN
\title{E214 \\ ATLAS}

\author{Christopher Deutsch\footnote{christopher.deutsch@uni-bonn.de} \and Christian Bespin\footnote{christian.bespin@uni-bonn.de}}

\date{\today}

\begin{document}

\begin{titlepage}

\maketitle

% DURCHFÜHRUNGSDATUM UND TUTOR
\begin{center}
\begin{tabular}{l r}
Durchführung: & 7./8. März 2016 \\
Gruppe: & P8 \\
Tutor: & Elisabeth Schopf
\end{tabular}
\end{center}

% ZUSAMMENFASSUNG
\begin{abstract}
\noindent Zusammenfassung hier
\end{abstract}

\end{titlepage}

% INHALTSVERZEICHNIS
\tableofcontents
% Neue Seite nach TOC
\newpage

% INHALT VERSUCHSPROTOKOLL
\section{Theorie}

\subsection{Einführung}

\subsubsection{Standardmodell}

Das Standardmodell beschreibt die Elementarteilchen und die Wechselwirkungen zwischen ihnen, mit Ausnahme der Gravitation und vereint somit die Erkenntnisse der Teilchenphysik.
Es gilt seit der Entdeckung des Higgs-Bosons als abgeschlossen.
Im Rahmen des Standardmodells können die einzelnen Elementarteilchen in Gruppen zu Leptonen, Quarks und Kraftteilchen zusammengefasst werden.
Leptonen und Quarks sind Fermionen, d.h. sie sind Spin~-$1/2$-Teilchen und unterscheiden sich durch ihre Ladung.
Während Leptonen, deren bekanntester Vertreter das Elektron ist, eine Ladungszahl~$1$ haben und die zugehörigen Neutrinos ungeladen sind, tragen Quarks drittelzahlige Ladungen~($1/3$ bzw.~$2/3$).
Durch Eichbosonen, wie Photon, Gluon, Z- und W-Bosonen wirken Kräfte zwischen den Quarks und Leptonen.
Das Higgs-Boson fällt aus dieser Einteilung heraus, da es eine Folge der spontanen Symmetriebrechung ist, die mit der Ursache von Masse assoziiert wird.

Durch diese Einteilung lässt sich eine Übersicht der einzelnen Elementarteilchen des Standardmodells gut wie in Abbildung \korr{REF} grafisch darstellen.

\subsubsection{LHC und ATLAS}

Der Large Hadron Collider (LHC) der europäischen Organisation für Kernforschung (CERN) ist der zur Zeit weltgrößte Teilchenbeschleuniger mit einer Schwerpunktsenergie von bis zu~\SI{14}{TeV}.
Dadurch können physikalische \korr{hethetheth mir fehlt ein Wort} bei hohen Energien überprüft und möglicherweise neue Beobachtungen gemacht werden.
Hierzu kommen vier Detektoren zum Einsatz, von denen zwei der Untersuchung hochenergetischer Ereignisse auf der Tera-Skala dienen.
Einer dieser Detektoren ist der ATLAS-Detektor, mit dem die in diesem Versuch analysierten Daten aufgenommen wurden.
\korr{Hier Beschreibung und Aufbau des Detektors?}

\subsection{Kinematik}

\subsubsection{Lorentzvektoren}
Ein Lorentzvektor~$\mathbf{x}$ ist ein vierdimensionaler Vektor mit reellen Komponenten, welche bezüglich einer Lorentztransformation~$\Lambda$ gemäß
\begin{align*}
	{x^\prime}^\mu = \tensor{\Lambda}{^\mu_\nu} x^\nu \qquad {x^\prime}_\mu = \tensor{\left(\Lambda^{-1}\right)}{^\nu_\mu} x_\nu
\end{align*}
transformieren.
Dadurch ist das Minkowski-Produkt zweier Lorentzvektoren
\begin{align*}
	\mathbf{x} \cdot \mathbf{y} = \tensor{g}{_\mu_\nu} x^\mu y^\nu \quad \text{mit} \quad g = \mathrm{diag}(1,-1,-1,-1)
\end{align*}
invariant unter Lorentztransformationen.

Ein solcher Lorentzvektor ist der sog.\ Viererimpuls~$\mathbf{p} = \left(E, \vec{p}\right)$, bestehend aus der Gesamtenergie~$E$ und dem Impuls~$\vec{p}$ im jeweiligen Bezugssystem.
Dieser ist essenziell für kinematische Betrachtungen in der Teilchenphysik und tritt häufig in Form von lorentzinvarianten Skalarprodukten auf, die in einem beliebigen Bezugssystem (z.B.\ im Schwerpunktssystem) ausgewertet werden können.

Beispiele dafür sind die invariante Masse eines Teilchens~$m_\mathrm{inv}^2 = \mathbf{p}^2$ oder die Mandelstam-Variable~$s = (\mathbf{p_1} + \mathbf{p_2})^2$, welche der quadratischen Schwerpunktsenergie zweier Teilchen entspricht.

\subsubsection{Rapidität und Pseudorapidität}
Die Rapidität entlang der Strahlachse~$z$ eines Teilchenbeschleunigers ist definiert als
\begin{align*}
	y = \frac{1}{2}\ln\left( \frac{E+p_z}{E-p_z} \right)
\end{align*}
mit der Energie~$E$ und dem Impuls~$p_z$ entlang der Strahlachse.
Wie das Lorentz-Beta~$\beta$ ist die Rapidität ein Maß für die Geschwindigkeit eines Teilchens.
Jedoch ist die Rapidität im Vergleich zu $\beta$ gegenüber Lorentz-Boosts entlang der Strahlachse lorentzinvariant und somit ein additives Geschwindigkeitsmaß, wie es bereits aus der klassischen Mechanik bekannt ist.

Für Teilchen mit verschwindender Masse ($E \approx p$) geht die Rapidität~$y$ in die Pseudorapidität~$\eta$ über
\begin{align}
	\eta = \frac{1}{2} \ln\left( \frac{1 + \cos\theta}{1 - \cos\theta} \right) = - \ln\left( \tan\frac{\theta}{2} \right)\text{,}
	\label{eq:pseudorapidity}
\end{align}
welche nur noch eine Funktion des Polarwinkels $\theta$ im jeweiligen Bezugssystem darstellt.
Im Gegensatz zu dem Polarwinkel~$\theta$ sind Pseudorapiditätsdifferenzen lorentzinvariant und somit ein bevorzugtes Maß für die Beschreibung von Ereignissen in Teilchendetektoren.
Beispielsweise eine Pseudorapidität von $\eta = \num{0}$ einem Polarwinkel von $\theta = \SI{90}{\degree}$ (senkrecht zur Strahlachse) und positive/negative Pseudorapiditäten entsprechen Winkeln in/entgegen der Richtung der Strahlachse.

\subsubsection{Hadronen-Kollision}
Bei Kollisionen an Hadronenbeschleunigern wie dem LHC gilt es zu beachten, dass die eigentliche Kollision nicht zwischen den Hadronen, sondern dessen Partonen stattfindet.
Im Partonmodell besteht jedes Hadron neben Valenzquarks auch aus Gluonen und paarerzeugten Quark-Antiquark-Paaren.
Jedes Parton trägt dabei einen Impulsbruchteil, welcher im Bezugssystem in dem das Hadron einen unendlichen longitudinalen Impuls hat (d.h.\ transversale Impulse können vernachlässigt werden) durch die Bjorken'sche Skalenvariable~$x$ gegeben ist.

Zur Charakterisierung von Hadronen-Kollisionen ist es demnach erforderlich die Wahrscheinlichkeitsverteilung, dass ein Parton den Impulsbruchteil~$x$ des Hadrons trägt, zu bestimmen.
Diese Verteilungen, welche man Partonverteilungsfunktionen~$f(x)$ nennt, wurden für die verschiedenen Partonentypen~($u, \bar{u}, d, \dots, g$) an mehreren Teilchenbeschleunigern vermessen.

Dennoch sind die spezifischen Impulsbruchteile~$x_1, x_2$ der kollidierenden Partonen in der Regel unbekannt, so dass der Transversalimpuls~$p_\mathrm{T}$, welcher beispielsweise im durch den Krümmungsradius geladener Teilchen im äußeren Magnetfeld bestimmt werden kann, für die Auswertung herangezogen wird.
Darüber hinaus kann aus der Summe der Transversalimpulse aller Teilchen im Endzustand, welche gemäß der Impulserhaltung für frontal zusammenstoßende Partonen zu null summieren müssen, fehlender Transversalimpuls reproduziert werden, der auf nicht-detektierbare Teilchen wie Neutrinos zurückzuführen ist.

Da der Transversalimpuls für ungeladene Teilchen wie beispielsweise Photonen nicht durch den Krümmungsradius im Magnetfeld vermessen werden kann, wird stattdessen auf die Energieeinträge in den Kalorimetern zurückgegriffen und die Größe der fehlenden transversalen Energie~$\slashed{E}_\mathrm{T}$
\begin{align*}
	\slashed{E}_\mathrm{T} = - \sum_{i} E^i \sin\theta_i \, \vec{n}_{i,\perp}
\end{align*}
mit den Energieeinträgen~$E_i$ des Kalorimeters unter Polarwinkeln~$\theta_i$  und azimuthalen Einheitsvektoren~$\vec{n}_{i,\perp}$ auf den jeweiligen Eintrag, definiert.



\subsection{Neue Physik}

\subsubsection{Higgs-Boson}

\subsubsection{Supersymmetrie}


\section{Beschreibung der Software (ist sowas wie Versuchsaufbau)}

\subsection{ATLANTIS}

Zur Darstellung von Ereignissen wird das Programm Atlantis verwendet, das den Aufbau des ATLAS Detektors zeigt und die jeweiligen Signale der einzelnen Subdetektoren für einzelne Ereignisse darstellt.
Das Hauptfenster ist dabei in vier Darstellungen geteilt, wie in Abbildung \ref{fig:atlantis} zu sehen ist.
Den größten Bildbereich nimmt eine Ansicht der $xy$-Ebene des Detektors ein, die den gesamten Querschnitt zeigt.
Zusätzlich wird oben rechts ebenfalls eine Ansicht des Querschnitts gezeigt, jedoch mit Fokus auf den inneren Detektor, in dem Trajektorien geladener Teilchen dargestellt werden können. Am rechten Rand des Fensters ist ein Schnitt des Detektors in Längstrichtung gezeigt, während im unteren Bereich eine Ausschnitt in dieser $\rho z$-Ebene gezeigt wird, der vor allem den Bereich des inneren Detektors und der Kalorimeter zeigt, aber das Myonsystem ausblendet.
Die Anzahl und Ansicht der einzelnen Fenster kann dabei individuell geändert werden je nach Information, die man aus einem Ereignis gewinnen möchte.
\begin{figure}[htbp]
	\centering
	\includegraphics[width=\textwidth]{./data/atlantis/atlantis_empty.png}
	\caption{Hauptfenster des Atlantis-Programms in der Standardansicht}
	\label{fig:atlantis}
\end{figure}

\subsection{ROOT}
Zur Bearbeitung der Versuchsaufgaben 2 bis 4 wird das Datenanalyseprogramm ROOT verwendet.
ROOT wird über einen C++-Interpreter aus der Kommandozeile heraus gesteuert und bietet vielfältige Möglichkeiten zur Datenanalyse.
Die für den Versuch nötigen Daten und Funktionen werden als C++-Objekte vor Durchführung der Aufgaben in ROOT geladen.


\section{Versuchsaufgabe 1: Darstellung von Teilchenreaktionen}

\subsection{Lerndatensätze}

Zu Beginn des Versuchs wurde der Umgang mit Atlantis geübt.
Hierzu wurden verschiedene Lerndatensätze betrachtet, die Ereignisse von Elektronen, Myonen, Photonen, $\tau$~-Leptonen und Jets enthalten.
Exemplarisch sollen für diese Teilchen im Folgenden einige Ereignisse vorgestellt werden, mit denen die Antwort des Detektors auf verschiedene Teilchenarten verstanden werden kann.

\clearpage
\subsubsection{Elektronen}
\begin{figure}[htbp]
	\centering
	\includegraphics[width=1.0\textwidth]{./data/atlantis/singlepart_events_new/electron/curvature.png}
	\caption{Signatur eines einzelnen Elektrons im ATLAS-Detektor}
	\label{fig:electron-curvature}
\end{figure}
\vfill
\noindent
Im Lerndatensatz der Elektronen finden sich Ereignisse in denen einzelne Elektronen erzeugt wurden.
In Abbildung \ref{fig:electron-curvature} wurde ein typisches Ereignis mit einem einzelnen Elektron aufgetragen.
Die Signatur eines Elektrons ist gekennzeichnet durch eine Trajektorie im inneren Detektor, welche aufgrund der elektromagnetischen Interaktion des Elektrons mit dem Detektormaterial als Ladungseintrag in den Halbleiter- und Übergangsstrahlungsdetektoren rekonstruiert werden kann.
Das Magnetfeld im inneren Detektor, welches in Richtung der Strahlröhre zeigt, zwingt die Elektronen dort auf einen Kreisbogen und wird zur Rekonstruktion von Ladung und transversalem Impuls genutzt.
Im Fall von Abbildung \ref{fig:electron-curvature} kann an der Krümmung abgelesen werden, dass es sich um ein negativ geladenes Elektron handelt (das Magnetfeld zeigt in der $xy$-Ansicht aus der Bildebene).
Schließlich bildet das Elektron im elektromagnetischen Kalorimeter eine Kaskade aus Photonen und Elektron-Positron-Paaren aufgrund von Bremsstrahlungs- und Paarbildungsprozessen.
\vfill

\clearpage
\begin{figure}[htbp]
	\centering
	\includegraphics[width=1.0\textwidth]{./data/atlantis/singlepart_events_new/electron/twopart.png}
	\caption{Signatur eines Elektrons mit zusätzlichen geladenen Spuren im ATLAS-Detektor}
	\label{fig:electron-twopart}
\end{figure}
\vfill
\noindent
In Abbildung \ref{fig:electron-twopart} wurden zwei geladene Spuren im inneren Detektor rekonstruiert.
Da die zugrundeliegenden Ereignisse jedoch nur aus einem einzelnen Elektron bestehen, muss die zweite Spur durch das Ereignis-Elektron im Detektormaterial erzeugt worden sein.
Eine Möglichkeit dafür ist die Erzeugung eines hoch-energetischen Delta-Elektrons durch das Primärelektron.
Die Wahrscheinlichkeit für die Emission eines Delta-Elektrons in Vorwärtsrichtung ist jedoch sehr klein, daher könnte eine Alternative die Erzeugung eines Bremsstrahlphotons und die folgliche Paarbildung sein, in der das dritte Elektron nicht im Detektor rekonstruiert werden konnte.
\vfill

\clearpage
\subsubsection{Myonen}
\begin{figure}[htbp]
	\centering
	\includegraphics[width=1.0\textwidth]{./data/atlantis/singlepart_events_new/muon/single_track2.png}
	\caption{Signatur eines einzelnen Myons im ATLAS-Detektor}
	\label{fig:myon-singletrack}
\end{figure}
\vfill
\noindent
In Abbildung \ref{fig:myon-singletrack} ist die typische Detektorantwort auf ein Myon zu sehen.
Da es sich bei Myonen um geladene Teilchen handelt, hinterlassen diese eine Spur im inneren Detektor und passieren das dort anliegende Magnetfeld auf einer gekrümmten Bahn in der $xy$-Ebene.
Sie durchtreten die beiden Kalorimeter beinahe ungehindert und deponieren dabei nur eine geringe Energiemenge.
Im Myonsystem, das mit zahlreichen Detektorkammern und dem Magnetfeld eines toroidalen Spulensystems ausgestattet ist, können ebenfalls Spuren rekonstruiert werden.
Aufgrund des Magnetfeldes werden Myonen jedoch in der $\rho z$-Ebene gekrümmt, was eine zweite unabhängige Impulsmessung ermöglicht.
Da andere hadronische oder leptonische Teilchen im elektromagnetischen oder hadronischen Kalorimeter gestoppt werden und Neutrinos keine Spuren in den Detektorkammern des Myonsystems hinterlassen, ist diese Teilchensignatur auf ein Myon zurückzuführen. 
\vfill

\clearpage
\subsubsection{Photonen}
\begin{figure}[htbp]
	\centering
	\includegraphics[width=1.0\textwidth]{./data/atlantis/singlepart_events_new/photons/single_photon_2.png}
	\caption{Signatur eines einzelnen Photons im ATLAS-Detektor}
	\label{fig:photon}
\end{figure}
\vfill
\noindent
Wie in Abbildung \ref{fig:photon} zu sehen, hinterlassen Photonen als ungeladene Teilchen keine rekonstruierbaren Spuren im inneren Detektor.
Da sie jedoch elektromagnetisch mit Materie interagieren können, formt sich im elektromagnetischen Kalorimeter ein Schauer aus Photonen und Elektron-Positron-Paaren (ähnlich zum Elektron).
In dieser Kaskade wird die gesamte Energie des Photons deponiert.
\vfill

\clearpage
\begin{figure}[htbp]
	\centering
	\includegraphics[width=1.0\textwidth]{./data/atlantis/singlepart_events_new/photons/conversion.png}
	\caption{Signatur eines $\ell^- \ell^+$-Paares aus dem Paarbildungsprozess eines einzelnen Photons im ATLAS-Detektor}
	\label{fig:photon-conversion}
\end{figure}
\vfill
\noindent
Im Ereignis in Abbildung \ref{fig:photon-conversion} entsteht ein Elektron-Positron-Paar durch Paarbildung des primären Photons im Detektormaterial.
Die Elektronen haben dabei entgegengesetzte Ladungen, sodass die Trajektorien im inneren Detektor in unterschiedliche Richtungen gekrümmt sind.
Darüber hinaus ist es möglich den gemeinsamen Vertex der Elektronen vom Interaktionsvertex zu unterscheiden.
Dies wurde in Abbildung \ref{fig:photon-vertex} mit der Vertexrekonstruktion von ATLANTIS dargestellt, sodass der sekundäre Vertex einen Abstand von etwa \SI{5}{\centi\meter} vom Interaktionsvertex hat.
\vfill
\begin{figure}[htbp]
	\centering
	\includegraphics[width=1.0\textwidth]{./data/atlantis/singlepart_events_new/photons/conversion_vertex_no_fisheye.png}
	\caption{Von ATLANTIS rekonstruierter Vertex (magenta) des Elektron-Positron-Paars in einer Entfernung von etwa \SI{5}{\centi\meter} vom Interaktionsvertex (rot).}
	\label{fig:photon-vertex}
\end{figure}

\clearpage
\subsubsection{$\tau$-Leptonen}
\begin{figure}[htbp]
	\centering
	\includegraphics[width=1.0\textwidth]{./data/atlantis/singlepart_events_new/tau/muon.png}
	\caption{Signatur des leptonischen Zerfalls eines $\tau^+$-Leptons im ATLAS-Detektor.}
	\label{fig:tau-muon}
\end{figure}
\vfill
\noindent
Das Ereignis in Abbildung \ref{fig:tau-muon} des $\tau$-Lerndatensatzes zeigt den Zerfall
\begin{align*}
	\tau^+ \rightarrow \mu^+ + \nu_\mu + \bar{\nu}_\tau \, \text{.}
\end{align*}
Aufgrund der kurzen Lebensdauer zerfällt das $\tau$-Lepton nach sehr kurzer Flugstrecke $c \tau = \SI{87}{\micro\meter}$ \cite{pdg}, weshalb lediglich das Myon im Detektor rekonstruiert werden kann.
Aufgrund der Leptonen-Universalität ist auch ein Zerfall in Elektron und zugehöriges Neutrino  mit ungefähr dem gleichen Verzweigungsverhältnis (abgesehen vom Phasenraumunterschied) möglich.
Ein solcher Zerfall würde lediglich eine geladene Spur und einen Schauer im elektromagnetischen Kalorimeter verursachen und nicht wie bei dem hier betrachteten Ereignis im Myonsystem registriert werden.

In beiden Fällen entstehen zusätzlich zu dem geladenen Lepton auch ein $\tau$-Neutrino und das jeweilige Neutrino des Elektrons bzw.\ Myons.
Diese können nicht detektiert werden, die fehlende transversale Energie (in den beiden oberen Ansichten der Abbildung \ref{fig:tau-muon} rot bzw.\ grau gestrichelt) ist jedoch ein Indiz für ihre Existenz.
\vfill

\clearpage
\begin{figure}[htbp]
	\centering
	\includegraphics[width=1.0\textwidth]{./data/atlantis/singlepart_events_new/tau/single_pion.png}
	\caption{Signatur eines hadronischen $\tau$-Zerfalls im ATLAS-Detektor}
	\label{fig:tau-pion}
\end{figure}
\vfill
\noindent
Ein Beispiel für einen hadronischen $\tau$-Zerfall liefert Abbildung \ref{fig:tau-pion}.
In diesem Fall ist eine geladene Spur im inneren Detektor sichtbar, sodass ein Zerfall mit einem geladenen Pion und einem $\tau$-Neutrino im Endzustand stattgefunden haben muss.
Weiterhin können noch bis zu drei neutrale Pionen auftreten, welche nicht im inneren Detektor sichtbar sind.
Aufgrund der Lokalisierung der Energie im hadronischen Kalorimeter in einem kleinen Raumbereich ist ein Zerfall in ein einzelnes geladenes Pion zu vermuten.
Es ist ebenfalls die fehlende transversale Energie aufgrund des $\tau$-Neutrinos zu beobachten, welche im Gegensatz zum leptonischen Zerfall nun ein Maß für den transversalen Impuls des Neutrinos darstellt, da nur ein Neutrino bei diesem Ereignis auftritt.
\vfill

\clearpage
\subsubsection{Zwei-Jet}
\begin{figure}[htbp]
	\centering
	\includegraphics[width=1.0\textwidth]{./data/atlantis/singlepart_events_new/jets/muon.png}
	\caption{Signatur eines Zwei-Jet Ereignisses im ATLAS-Detektor}
	\label{fig:jets-muon}
\end{figure}
\vfill
\noindent
In dem in Abbildung \ref{fig:jets-muon} zugrundeliegenden Ereignis entstehen zwei Quarks oder Gluonen, welche folglich aufgrund des \textit{Confinements} zwischen farb-geladenen Teilchen in Mesonen und Baryonen hadronisieren.
Die so entstehenden geladenen Hadronen bilden im inneren Detektor eine Ansammlung vieler Spuren in einem engen Konus.
Ein solcher \textit{Jet} enthält sowohl geladene als auch ungeladene Pionen.
Die geladenen Pionen haben eine Zerfallslänge in der Ordnung von $c \tau = \SI{7.8}{\meter}$~\cite{pdg} und können dadurch direkt in den hadronischen Kalorimetern beobachtet werden.
Im Gegensatz dazu zerfallen die ungeladenen Pionen schnell in zwei Photonen ($c \tau = \SI{25.5}{\nano\meter}$~\cite{pdg}) und führen somit ebenfalls zu signifikanten Einträgen in den elektromagnetischen Kalorimetern.

Das Myon, welches im oberen Jet in Abbildung \ref{fig:jets-muon} zu erkennen ist, ist ein Indikator dafür, dass das hadronisierende Quark-Paar ein $b\bar{b}$-Paar gewesen ist, da der Zerfall des im Jet entstandenen B-Mesons zu einem geladenen Lepton im Endzustand führen kann.
\vfill

\subsection{Energieverlust von Myonen in den Kalorimetern des ATLAS-Detektors}
Im Folgenden soll anhand der ersten 20 Myonen-Ereignisse aus den Lerndatensätzen der mittlere Energieverlust der Myonen in den Kalorimetern bestimmt werden. 

\subsubsection{Bestimmung des Myonenimpulses mit Atlantis}
\label{sssec:myon_momenta}
Der ATLAS-Detektor besitzt mit dem Magnetfeld des Solenoids im inneren Detektor und dem des toroidalen Spulensystems im Myonensystem zwei unabhängige Methoden um den Myonenimpuls zu bestimmen.
Bei den betrachteten Ereignissen können mithilfe der Ereignisanzeige Atlantis die Parameter (Impuls~$p$, transversaler Impuls~$p_\mathrm{T}$, Pseudorapidität~$\eta$, \dots) der Myonen jeweils im inneren Detektor und im Myonensystem abgefragt werden.

Um den Energieverlust der Myonen im Kalorimeter zu bestimmen, ist es nötig den Impuls~$p$ ebendieser zu bestimmen.
Da Atlantis auf diesen Parameter jedoch keinen Fehler angibt, wurde stattdessen der transversale Impuls~$p_\mathrm{T}$ und die Pseudorapidität~$\eta$ genutzt, um die Fehlerfortpflanzung per Hand durchzuführen.
Nach der Definition in Gleichung \eqref{eq:pseudorapidity} und $p = p_\mathrm{T} / \sin\vartheta$ kann der Impuls gemäß
\begin{align*}
	p = p_\mathrm{T} \cosh(\eta)
\end{align*}
berechnet werden.
Da der relative Fehler der Pseudorapidität gemäß Atlantis in der Größenordnung $10^{-3}$ liegt, soll dieser im Folgenden vernachlässigt werden.
Dann folgt aus \textsc{Gauß}scher Fehlerfortpflanzung für den Fehler des Impulses
\begin{align*}
	\sigma_p = | \cosh(\eta) \cdot \sigma_{p_\mathrm{T}} | \, \text{.}
\end{align*}
Die Messdaten sowie die Berechnung des Impulses~$p$ der Myonen im inneren Detektor und des Impulses~$p^\prime$ im Myon-Detektor wurde in Tabelle \ref{tab:muon_momenta} zusammengetragen.

Bei dem 10.\ Ereignis musste festgestellt werden, dass keine Trajektorie im Myon-Detektor rekonstruiert werden konnte.
Der Vergleich mit dem Pfad des Myons im inneren Detektor legt dabei nahe, dass dieses durch einen unsensitiven Bereich des Detektors zwischen Barrel und Endkappe entkommen konnte.
Daher wurde zusätzlich das 21.\ Ereignis vermessen.


\begin{sidewaystable}
	\centering
	\begin{tabular}{rSSSSSSSSSS}
\toprule
{Ereignis} & {$p_T$ / \si{\GeV}} & {$\sigma_{p_T}$ / \si{\GeV}} & {$\eta$} & {$p$ / \si{\GeV}} & {$\sigma_p$ / \si{\GeV}} & {$p_T^\prime$ / \si{\GeV}} & {$\sigma_{p_T^\prime}$ / \si{\GeV}} & {$\eta^\prime$} & {$p^\prime$ / \si{\GeV}} & {$\sigma_{p^\prime}$ / \si{\GeV}} \\
\midrule
         1 &               -38.4 &                          1.0 &     1.44 &             -85.3 &                      2.3 &                      -24.1 &                                 2.6 &            1.45 &                    -53.9 &                               5.8 \\
         2 &                33.2 &                          0.8 &    -0.77 &              43.4 &                      1.0 &                       33.4 &                                 1.7 &           -0.77 &                     43.8 &                               2.2 \\
         3 &               -41.9 &                          3.6 &    -2.44 &            -241.4 &                     21.0 &                      -41.0 &                                 0.7 &           -2.44 &                   -236.9 &                               3.8 \\
         4 &                42.0 &                          0.8 &     0.57 &              48.9 &                      1.0 &                       38.1 &                                 1.3 &            0.58 &                     44.8 &                               1.6 \\
         5 &               -53.7 &                          2.1 &    -1.81 &            -168.2 &                      6.4 &                      -61.2 &                                 4.4 &           -1.73 &                   -177.6 &                              12.8 \\
         6 &                44.6 &                          1.4 &     1.62 &             117.3 &                      3.6 &                       36.6 &                                 3.1 &            1.62 &                     96.5 &                               8.3 \\
         7 &               -57.3 &                          1.7 &     0.70 &             -71.9 &                      2.1 &                      -51.8 &                                 1.1 &            0.70 &                    -65.0 &                               1.3 \\
         8 &                64.5 &                          2.7 &     1.80 &             200.0 &                      8.3 &                       64.9 &                                 2.8 &            1.79 &                    199.3 &                               8.7 \\
         9 &               -55.5 &                          1.1 &    -0.29 &             -57.8 &                      1.2 &                      -48.0 &                                 0.9 &           -0.27 &                    -49.7 &                               1.0 \\
        10 &                39.9 &                          1.0 &     1.18 &              71.1 &                      1.7 &                            &                                     &                 &                          &                                   \\
        11 &               -37.8 &                          1.2 &    -1.64 &            -100.7 &                      3.2 &                      -35.1 &                                 0.7 &           -1.65 &                    -94.3 &                               1.8 \\
        12 &                37.6 &                          0.7 &     0.19 &              38.3 &                      0.7 &                       33.9 &                                 1.2 &            0.19 &                     34.5 &                               1.2 \\
        13 &               -59.9 &                          1.8 &     1.16 &            -105.2 &                      3.1 &                      -62.1 &                                 4.2 &            1.16 &                   -108.7 &                               7.3 \\
        14 &                47.5 &                          3.3 &     2.29 &             236.1 &                     16.5 &                       53.0 &                                 1.2 &            2.29 &                    263.5 &                               5.9 \\
        15 &               -43.9 &                          1.5 &    -1.76 &            -131.7 &                      4.4 &                      -42.0 &                                 2.1 &           -1.76 &                   -125.5 &                               6.3 \\
        16 &                45.7 &                          1.8 &    -1.87 &             152.2 &                      6.1 &                       47.3 &                                 2.2 &           -1.88 &                    157.7 &                               7.2 \\
        17 &               -32.6 &                          0.5 &     0.40 &             -35.2 &                      0.6 &                      -29.8 &                                 0.5 &            0.40 &                    -32.2 &                               0.6 \\
        18 &                52.8 &                          1.1 &     0.23 &              54.2 &                      1.1 &                       48.7 &                                 2.3 &            0.23 &                     50.0 &                               2.3 \\
        19 &               -64.5 &                          2.5 &     0.77 &             -84.7 &                      3.3 &                      -52.2 &                                 1.0 &            0.76 &                    -68.1 &                               1.4 \\
        20 &                47.6 &                          1.3 &     1.42 &             104.2 &                      2.9 &                       49.5 &                                 3.2 &            1.42 &                    108.0 &                               7.0 \\
        21 &               -35.5 &                          2.3 &    -2.33 &            -184.1 &                     11.8 &                      -33.2 &                                 0.8 &           -2.32 &                   -170.8 &                               4.0 \\
\bottomrule
\end{tabular}

	\caption{Bestimmung des Impulses des Myons im inneren Detektor (ungestrichene Größen) und im Myonensystem (gestrichene Größen) mithilfe der gemessenen Pseudorapidität~$\eta$ und des transversalen Impulses~$p_\mathrm{T}$.
		Für Ereignis 10 konnte kein äußeres Myon rekonstruiert werden (für weitere Erklärung siehe Abschnitt \ref{sssec:myon_momenta}) stattdessen wurde noch das 21.\ Ereignis in die Analyse einbezogen.
		Das Vorzeichen von ~$p_\mathrm{T}$ dient der Unterscheidung positiv und negativ geladener Teilchen.}
	\label{tab:muon_momenta}
\end{sidewaystable}

\subsubsection{Bestimmung von Impuls- und Energieverlust der Myonen im Kalorimeter}
Nachfolgend soll der Impuls- bzw.\ Energieverlust der vermessenen Myonen im Kalorimeter berechnet werden.
Zunächst betrachte man, dass der Myonenimpuls~$p^{(\prime)}$ in der Größenordnung von einigen \SI{10}{\GeV} liegt.
Im Vergleich dazu beträgt die Masse der Myonen
\begin{align*}
	m_\mu \approx \SI{105.658}{\MeV} \, \cite{pdg} \text{,}
\end{align*}
weshalb $p \gg m_\mu$ ist und dadurch in guter Approximation $p^{(\prime)} = E^{(\prime)}$ angenommen werden kann.

Anschließend kann der Impuls- und Energieverlust aus der Differenz von Myon-Impuls im inneren Detektor~$p$ und Myon-Impuls im Myon-Detektor~$p^\prime$ berechnet werden:
\begin{align}
	\Delta E \approx \Delta p = p - p^\prime
	\label{eq:p_diff}
\end{align}
und der Fehler nach \textsc{Gauß}scher Fehlerfortpflanzung
\begin{align*}
	\sigma_{\Delta E} = \sqrt{\sigma_p^2 + \sigma_{p^\prime}^2} \, \text{.}
\end{align*}
Die somit berechneten Energieverluste wurden in Tabelle \ref{tab:energy_loss_muons} aufgetragen.
Anzumerken ist, dass für die Auswertung von Gleichung \eqref{eq:p_diff} das Ladungsvorzeichen, welches von Atlantis auf alle Impulse angegeben wird, verworfen wurde.

\begin{table}[h]
	\centering
	\begin{subtable}{.5\textwidth}
		\centering		
		\begin{tabular}{rSS}
\toprule
{Ereignis} & {$\Delta E$ / \si{\GeV}} & {$\sigma_{\Delta E}$ / \si{\GeV}} \\
\midrule
         1 &                     31.4 &                               6.2 \\
         2 &                     -0.4 &                               2.4 \\
         3 &                      4.5 &                              21.3 \\
         4 &                      4.1 &                               1.8 \\
         5 &                     -9.4 &                              14.3 \\
         6 &                     20.8 &                               9.1 \\
         7 &                      7.0 &                               2.5 \\
         8 &                      0.7 &                              12.1 \\
         9 &                      8.1 &                               1.5 \\
        11 &                      6.4 &                               3.6 \\
\bottomrule
\end{tabular}
	
	\end{subtable}%
	\begin{subtable}{.5\textwidth}
		\centering
		\begin{tabular}{rSS}
\toprule
{Ereignis} & {$\Delta E$ / \si{\GeV}} & {$\sigma_{\Delta E}$ / \si{\GeV}} \\
\midrule
        12 &                      3.8 &                               1.4 \\
        13 &                     -3.5 &                               8.0 \\
        14 &                    -27.4 &                              17.5 \\
        15 &                      6.2 &                               7.7 \\
        16 &                     -5.5 &                               9.5 \\
        17 &                      3.1 &                               0.8 \\
        18 &                      4.2 &                               2.6 \\
        19 &                     16.7 &                               3.6 \\
        20 &                     -3.8 &                               7.6 \\
        21 &                     13.3 &                              12.5 \\
\bottomrule
\end{tabular}

	\end{subtable}
	\caption{Energieverlust~$\Delta E$ der Myonen im Kalorimeter des ATLAS-Detektors. Negative $\Delta E$ entsprechen einer Energiezunahme.}
	\label{tab:energy_loss_muons}
\end{table}

\subsubsection{Energieverlust und dessen Abhängigkeiten}
Betrachtet man den Energieverlust der Myonen in Tabelle \ref{tab:energy_loss_muons}, so stellt man fest, dass bei einigen Ereignissen ein negativer Energieverlust gemessen wurde.
Dies entspricht einem Energiegewinn der Myonen in den Kalorimetern und ist unphysikalisch.
Betrachtet man jedoch den Fehler dieser Größen, sind die meisten dieser unphysikalischen Myonen noch mit positiven Energieverlusten kompatibel.

Im Folgenden sollen die Abhängigkeiten des Energieverlustes von den kinematischen Variablen der Myonen untersucht werden.
Zunächst würde man erwarten, dass eine Abhängigkeit von der Pseudorapidität~$\eta$ existiert, da die Weglängen der Myonen durch die Kalorimeter bei unterschiedlichen Polarwinkeln~$\theta$ verschieden sind und somit im Mittel unterschiedliche Energieverluste zu erwarten sind.
Explizit würde man eine Zunahme des Energieverlustes mit steigender Pseudorapidität erwarten.
In Abbildung \ref{fig:muon_eloss_eta} wurde der Energieverlust gegen den Betrag der Pseudorapidität aufgetragen, welche zeigt, dass keine starke Abhängigkeit festgestellt werden kann.
Die Gründe dafür folgen nach dem nächsten Paragraphen.

Darüber hinaus ist eine starke Abhängigkeit von der Energie der Myonen zu erwarten, wenn diese größer ist als die kritische Energie\footnote{Myonenenergie mit gleichem Beitrag zum mittleren Energieverlust durch Ionisation und Bremsstrahlung: $\langle \frac{\mathrm{d}E}{\mathrm{d}x} \rangle_\mathrm{Ion.}(E_{\mu\mathrm{c}}) = \langle \frac{\mathrm{d}E}{\mathrm{d}x} \rangle_\mathrm{Brems.}(E_{\mu\mathrm{c}})$}~$E_{\mu\mathrm{c}}$ der Myonen im Kalorimetermaterial.
Da diese jedoch in der Größenordnung von mehreren \SI{100}{\GeV} \cite{pdg} liegt und der größte vermessene Impuls lediglich \SI{200}{\GeV} beträgt, sollte diese Abhängigkeit nicht in Erscheinung treten.
Zum Vergleich wurde auch hier die Abhängigkeit in Abbildung \ref{fig:muon_eloss_momentum} dargestellt.
\begin{figure}[hp]
	\begin{subfigure}{1.0\textwidth}
		\centering
		\includegraphics{./figures/muon_energy_loss/eta.pdf}
		\subcaption{}
		\label{fig:muon_eloss_eta}
	\end{subfigure}
	\begin{subfigure}{1.0\textwidth}
			\centering
			\includegraphics{./figures/muon_energy_loss/momentum.pdf}
			\subcaption{}
			\label{fig:muon_eloss_momentum}
	\end{subfigure}
	\caption{Abhängigkeiten des Energieverlustes der Myonen in den Kalorimetern des ATLAS-Detektors von der Pseudorapidität~$\eta$ des Myons und dessen Impuls~$p$. Ebenfalls aufgetragen ist eine Anpassung des mittleren Energieverlustes~$\langle \Delta E \rangle$.}
	\label{fig:muon_eloss}
\end{figure}

In beiden Fällen ist in Anbetracht der Fehler keine signifikante Abhängigkeit zu erkennen.
Im Wesentlichen liegt dies daran, dass der Energieverlust von Teilchen in Materie ein statistischer Prozess ist und man lediglich eine Distribution für diesen angeben kann (Laundau-Distribution).
Um eine bessere Bestimmung der Abhängigkeiten durchzuführen, sollten mehrere Energieverluste bei der gleichen zugrundeliegenden Größe (z.B.\ $\eta$, $p$, \dots) gemessen werden, um aus diesen einen Mittelwert zu bilden.

Schließlich soll der mittlere Energieverlust~$\Delta E$ der Myonen in den Kalorimetern berechnet werden.
Da keine signifikanten Abhängigkeiten aus den oben genannten Gründen festgestellt werden konnte, wird der mittlere Energieverlust durch eine $\chi^2$-Anpassung einer konstanten Funktion an die Daten durchgeführt.
Dadurch erhält man
\begin{align*}
	\langle \Delta E \rangle = \SI{4.46 +- 0.92}{\GeV}
\end{align*}
mit einem reduzierten $\chi$-Quadrat $\chi_\mathrm{red.}^2 = \num{3.11}$; das Ergebnis wurde ebenfalls in den Abbildungen \ref{fig:muon_eloss} aufgetragen.
Das reduzierte $\chi$-Quadrat spricht für keine gute Übereinstimmung der Fithypothese mit den gemessenen Daten, was jedoch nicht verwunderlich ist, da wie bereits erwähnt, der Energieverlust der Myonen einer statistischen Distribution folgt.

\subsection{Invariante Elektron-Positron-Masse im Prozess~$Z^0 \rightarrow e^+ + e^-$}
Im Folgenden sollen aus einem Datensatz von Zerfällen~$Z^0 \rightarrow e^+ + e^-$ drei Ereignisse gewählt werden und aus den rekonstruierten Elektronen die invariante Elektron-Positron-Masse bestimmt werden.
Bei der Auswahl der Ereignisse wurden die folgenden Bedingung gestellt:
\begin{itemize}
	\item zwei Elektronen mit hohem transversalen Impuls~$p_\mathrm{T}$
	\item eindeutig zuzuordnende Schauer im elektromagnetischen Kalorimeter
	\item entgegengesetzt Ladung beider Elektronen
\end{itemize}
Gemäß dieser Bedingungen wurden die Ereignisse 23, 33 und 37 des Datensatzes für die Auswertung ausgewählt.
Um die invariante Masse dieses Elektronen-Paares zu bestimmen, ist es notwendig den vektoriellen Impuls $\vec{p}$ beider Teilchen zu bestimmen.
Dazu wurde mithilfe von Atlantis der transversale Impuls~$p_\mathrm{T}$, die Pseudorapidität~$\eta$ und der Azimuthalwinkel~$\phi$ der rekonstruierten Elektronen bestimmt.
Die invariante Masse~$m_\mathrm{inv}$ kann dann gemäß
\begin{align*}
	m_\mathrm{inv}=\sqrt{(\mathbf{p}_{\mathrm{e}^-}+\mathbf{p}_{\mathrm{e}^+})^2}
\end{align*}
aus den beiden Vierervektoren von Elektron und Positron,~$\mathbf{p}_{\mathrm{e}^-}$ bzw.~$\mathbf{p}_{\mathrm{e}^+}$ berechnet werden.
Man erhält somit als Ausdruck für die invariante Masse
\begin{align}
	m_\mathrm{inv}&=\sqrt{2 m^2_{\mathrm{e}} + 2 ( E_{\mathrm{e}^-} E_{\mathrm{e}^+} - \vec{p}_{\mathrm{e}^-}\vec{p}_{\mathrm{e}^+} )} \nonumber \\
	&=\sqrt{ 2m^2_{\mathrm{e}} + 2 (E_{\mathrm{e}^-} E_{\mathrm{e}^+} - |\vec{p}_{\mathrm{e}^-}||\vec{p}_{\mathrm{e}^+}|\cos\theta)}\,\text{,}
	\label{eq:inv_mass}
\end{align}
wobei $\theta$ der Winkel zwischen den Impulsen beider Teilchen ist.
Da nur der Betrag des Impulses eines Teilchens gemessen werden kann, werden die Winkelinformationen~$\phi$ und~$\eta$ benötigt, um auf~$\theta$ zu schließen.
Dazu wird zunächst der Polarwinkel~$\vartheta$ im Laborsystem durch
\begin{align*}
	\vartheta = 2 \arctan\left( \mathrm{e}^{-\eta} \right)
\end{align*}
berechnet (folgt aus Gleichung \ref{eq:pseudorapidity}).
Anschließend können mit den Winkeln~$\phi$ und $\vartheta$ des sphärischen Koordinatensystems Einheitsvektoren in die Richtung der Elektronenimpulse durch
\begin{align*}
	\hat{p}_x = \sin\vartheta \cos\phi \qquad
	\hat{p}_y = \sin\vartheta \sin\phi \qquad
	\hat{p}_z = \cos\vartheta
\end{align*}
konstruiert werden.
Der Kosinus des Winkels zwischen dem Elektron- und Positronimpuls folgt dann aus dem Skalarprodukt
\begin{align*}
	\cos \theta = \hat{\vec{p}}_{\mathrm{e}^-} \cdot \hat{\vec{p}}_{\mathrm{e}^+}
\end{align*}
Bei der Berechnung dieses Werts kann aufgrund der kleinen Fehler in $\eta$ und $\phi$ der Fehler in~$\cos\theta$ vernachlässigt werden.
Außerdem muss noch der Betrag des Impulses~$p$ aus dem transversalen Impuls~$p_\mathrm{T}$ berechnet werden.
Dabei wird analog zu Abschnitt~\ref{sssec:myon_momenta} vorgegangen und wird hier nicht erneut zusammengetragen.

Nun soll die invariante Masse exakt ($m_\mathrm{e} = \SI{511}{\keV}$) und unter der Annahme verschwindender Elektronenmasse berechnet werden.
Bei der Berechnung dieser treten Terme der Form~$\sqrt{m_\mathrm{e}^2 + p^2}$ auf, wobei $p$ in der Größenordnung von einigen \SI{10}{\GeV} liegt.
Daher ist die Näherung der verschwindenden Elektronenmasse wegen
\begin{align*}
	1 - \frac{\sqrt{(\SI{1}{\MeV})^2 + (\SI{10}{\GeV})^2}}{\SI{10}{\GeV}} \sim 10^{-8}
\end{align*}
in Anbetracht typischer Messfehler identisch mit der exakten Rechnung.
Aus diesem Grund soll im Folgenden nur die Näherungsrechnung näher diskutiert werden.
Dennoch wurde die exakte Rechnung durchgeführt, welche keine relevante Abweichung von der Näherung zeigt.

Für die Näherung entspricht der Impuls der Elektronen deren Energie und somit folgt aus Gleichung~\eqref{eq:inv_mass}
\begin{align*}
	m_\mathrm{inv} = \sqrt{2 \, p_{\mathrm{e}^-} p_{\mathrm{e}^+} (1 - \cos\theta)}
\end{align*}
und aus \textsc{Gauß}scher Fehlerfortpflanzung
\begin{align*}
	\sigma_{m_\mathrm{inv}} = \sqrt{\frac{1 - \cos\theta}{2} \left( \frac{p_{\mathrm{e}^+}}{p_{\mathrm{e}^-}} \, \sigma_{p_{\mathrm{e}^-}}^2 + \frac{p_{\mathrm{e}^-}}{p_{\mathrm{e}^+}}  \, \sigma_{p_{\mathrm{e}^+}}^2\right)} \, \text{.}
\end{align*}
Die so berechneten Größen wurden in Tabelle \ref{tab:zee_inv_mass} zusammengetragen.

\begin{sidewaystable}[p!]
	\centering
	\begin{subtable}{0.585\textheight}
		\begin{tabular}{rSSSSSS}
\toprule
{Ereignis} & {$p_{\mathrm{T}^{(1)}}$ / \si{\GeV}} & {$\sigma_{p_\mathrm{T}^{(1)}}$ / \si{\GeV}} & {$\eta^{(1)}$} & {$\phi^{(1)}$ / \si{\degree}} & {$p^{(1)}$ / \si{\GeV}} & {$\sigma_{p^{(1)}}$ / \si{\GeV}} \\
\midrule
        33 &                                32.89 &                                        1.40 &           0.63 &                          10.9 &                    39.6 &                              1.7 \\
        37 &                                -8.71 &                                        0.34 &          -2.13 &                         262.2 &                   -37.1 &                              1.5 \\
        47 &                                45.51 &                                        2.78 &          -1.03 &                         204.8 &                    72.1 &                              4.4 \\
\bottomrule
\end{tabular}

		\subcaption{kinematische Größen der Positronen: transversaler Impuls~$p_\mathrm{T}^+$, Pseudorapidität~$\eta^+$, Azimuthalwinkel~$\phi^+$, Impulsbetrag~$p^+$ sowie zugehörige Standardfehler~$\sigma$}
		\vspace{1cm}
	\end{subtable}
	
	\begin{subtable}{0.585\textheight}
		\begin{tabular}{rSSSSSS}
\toprule
{Ereignis} & {$p_{\mathrm{T}^{-}}$ / \si{\GeV}} & {$\sigma_{p_\mathrm{T}^{-}}$ / \si{\GeV}} & {$\eta^{-}$} & {$\phi^{-}$ / \si{\degree}} & {$p^{-}$ / \si{\GeV}} & {$\sigma_{p^{-}}$ / \si{\GeV}} \\
\midrule
        33 &                             -41.72 &                                      0.79 &        -0.63 &                       195.6 &                 -50.3 &                            1.0 \\
        37 &                              -8.71 &                                      0.34 &        -2.13 &                       262.2 &                 -37.1 &                            1.5 \\
        47 &                             -32.08 &                                      2.11 &        -1.14 &                        31.6 &                 -55.1 &                            3.6 \\
\bottomrule
\end{tabular}

		\subcaption{kinematische Größen der Elektronen: transversaler Impuls~$p_\mathrm{T}^-$, Pseudorapidität~$\eta^-$, Azimuthalwinkel~$\phi^-$, Impulsbetrag~$p^-$ sowie zugehörige Standardfehler~$\sigma$}
		\vspace{1cm}
	\end{subtable}
	
	\begin{subtable}{0.585\textheight}
		\begin{tabular}{rSSSSS}
\toprule
{Ereignis} & {$\cos\theta$} & {$m_\mathrm{inv}^\mathrm{exakt}$ / \si{\GeV}} & {$\sigma_{m_\mathrm{inv}^\mathrm{exakt}}$ / \si{\GeV}} & {$m_\mathrm{inv}$ / \si{\GeV}} & {$\sigma_{m_\mathrm{inv}}$ / \si{\GeV}} \\
\midrule
        23 &         -0.773 &                                          82.6 &                                                1.6 &                           82.6 &                                     1.6 \\
        33 &         -0.998 &                                          89.2 &                                                2.1 &                           89.2 &                                     2.1 \\
        37 &         -0.969 &                                          80.1 &                                                2.0 &                           80.1 &                                     2.0 \\
\bottomrule
\end{tabular}

		\subcaption{Kosinus des Winkels zwischen den Elektron-Impulsen und die daraus folgenden invarianten Massen unter Beachtung des Elektronenmasse $m_\mathrm{inv}^\mathrm{exakt}$ und unter der Vernachlässigung der Masse $m_\mathrm{inv}$}
	\end{subtable}
	
	\caption{Messdaten und berechnete Größen zur Bestimmung der invarianten Elektron-Positron Masse.}
	\label{tab:zee_inv_mass}
\end{sidewaystable}

\section{Versuchsaufgabe 2: Kalibration der Elektronen}

Für die weitere Durchführung des Versuchs ist es sowohl in Aufgabe 3 als auch in Aufgabe 4 nötig, die Energie von Elektronen zu bestimmen.
Da einerseits die Kalorimeter aus vielen einzelnen Modulen bestehen und andererseits die Elektronen im inneren Detektor Energie verlieren oder in einen toten Bereich des Kalorimeters eintreten, ist die dort gemessene Energie systematisch zu gering.
Aus diesem Grund muss in einem ersten Schritt eine Energiekalibration des gesamten Detektors durchgeführt werden, wobei sowohl Position der Elektronen im Detektor als auch deren Eigenschaften berücksichtigt werden sollten.

Hierzu kommen Datensätze vom Zerfall $Z^0 \rightarrow e^+ + e^-$ zum Einsatz, da das $Z^0$ sehr genau gemessen wurde.
Der bekannte Peak des Zerfalls kann damit zur Kalibration des Energiesignals genutzt werden.
Die theoretische Erwartung der Verteilungsfunktion wird in guter Näherung auf eine nicht-relativistischen Breit-Wigner-Funktion mit Untergrund reduziert, die mit einer Gauss-Funktion zu einem Voigt-Profil gefaltet wird, um die (als gaußförmig angenommene) Verschmierung aufgrund der Detektorauflösung zu berücksichtigen.

Das  in ROOT geladene Fit-Objekt passt somit die Lage des Peaks, also Masse des $Z^0$, die Verbreiterung aufgrund der Detektorauflösung, den Untergrund (gegeben durch vier Parameter) und Normierung des Peaks und Untergrunds an.

\subsection{Aufgabe a) Darstellung kinematischer Variablen}

Zunächst werden mit dem Befehl \texttt{tree->Draw()} Histogramme der Elektronen- bzw Positronenenergie sowie der nvarianten Masse des Elektron-Positron-Paares erstellt.
Diese sind in den Abbildungen \ref{fig:el_energy} bis \ref{fig:inv_masse} zu finden.
Man kann bereits gut abschätzen, dass Elektron und Positron den gleichen Impulsbruchteil tragen, da das Maximum der Histogramme ungefähr bei der gleichen Energie in Höhe der halben Z-Masse auftritt.
Das Maximum der invarianten Elektron-Positron-Masse liegt bereits wie erwartet in der Größenordnung der $Z^0$-Masse.
Eine quantitative Aussage über die Lage des Z-Peaks liefert der Aufruf der Fit-Methode auf die unkalibrierten Daten, was in Abbildung \ref{fig:uncalibrated} dargestellt ist.
\begin{figure}[htbp]
	\centering
	\includegraphics[width=\textwidth]{./data/root/calibration/new/uncalibrated.pdf}
	\caption{Aufruf der Fit-Methode auf die unkalibrierten Daten. Die so gemessene Z-Masse beträgt \SI{85.8}{GeV}.}
	\label{fig:uncalibrated}
\end{figure}
\begin{figure}[htbp]
	\centering
	\begin{subfigure}{\textwidth}
		\centering
		\includegraphics[width=0.7\textwidth]{./data/root/calibration/new/el_energy.pdf}
		\subcaption{Histogrammierte Energie des Elektrons im $Z^0 \rightarrow e^+ + e^-$ Zerfall}
		\label{fig:el_energy}
	\end{subfigure}
	\begin{subfigure}{\textwidth}
		\centering
		\includegraphics[width=0.7\textwidth]{./data/root/calibration/new/pos_energy.pdf}
		\subcaption{Histogrammierte Energie des Positrons im $Z^0 \rightarrow e^+ + e^-$ Zerfall}
		\label{fig:pos_energy}
	\end{subfigure}
	\begin{subfigure}{\textwidth}
		\centering
		\includegraphics[width=0.7\textwidth]{./data/root/calibration/new/inv_mass.pdf}
		\subcaption{Histogrammierte invariante Masse des Elektron-Positron-Paars im $Z^0 \rightarrow e^+ + e^-$ Zerfall}
		\label{fig:inv_masse}
	\end{subfigure}
\end{figure}


\FloatBarrier
% BIBLIOGRAPHIE
\vspace{\fill}
% Maximale Anzahl der Einträge in Klammer
% Zitieren mit \cite{lamport94}
\begin{thebibliography}{19}
\bibitem{pdg}
	K.A. Olive \textit{et al.} (Particle Data Group),
	\emph{The Review of Particle Physics},
	Chin. Phys. C, \textbf{38}, 090001 (2014).
	
\bibitem{leo}
	W. R. Leo,
	\emph{Techniques for Nuclear and Particle Physics Experiments},
	Springer 1994

\end{thebibliography}

% APPENDIX
\begin{appendix}
\newpage
\section{Anhang}
\subsection{Vorbereitungsaufgaben}

\subsubsection{6.2.2 Frage A}
Der Viererimpuls muss in der Reaktion erhalten sein, so dass man schreiben kann:
\begin{align}
	\mathbf{p}_{Z^0}^2=m_{Z^0}^2&=(\mathbf{p}_{\mathrm{e}^-}+\mathbf{p}_{\mathrm{e}^+})^2\\
	&=2m^2_{\mathrm{e}} + 2 E_{\mathrm{e}^-} E_{\mathrm{e}^+} - 2\vec{p}_{\mathrm{e}^-}\vec{p}_{\mathrm{e}^+}
\end{align}
Im Ruhesystem des $Z^0$ gilt $\vec{p}_{\mathrm{e}^-}=-\vec{p}_{\mathrm{e}^+}$ und $E_{\mathrm{e}^-}=E_{\mathrm{e}^+}$. Daraus folgt:
\begin{align}
	m_{Z^0}^2&=2m^2_{\mathrm{e}} + 2 E_{\mathrm{e}}^2 + 2|\vec{p}_{\mathrm{e}}|^2
\end{align}
Aus der Energie-Impuls-Beziehung $E=\sqrt{m^2+|\vec{p}|^2}$ folgt somit:
\begin{align}
	m_{Z^0}^2 - 4m^2_{\mathrm{e}} &= 4|\vec{p}_{\mathrm{e}}|^2\\
	|\vec{p}_{\mathrm{e}}|&=\sqrt{\frac{m_{Z^0}^2 - 4m^2_{\mathrm{e}}}{4}} \approx \SI{45.6}{GeV}
\end{align}

\subsubsection{6.2.2 Frage B}
Die Aufgabe erfolgt analog zu Frage A, jedoch mit der Schwerpunktsenergie von \SI{5}{GeV} anstatt $m_{Z^0}$. Dann ergibt sich:
\begin{align}
	|\vec{p}_{\mathrm{\tau}}|&=\sqrt{\frac{(\SI{5}{GeV})^2 - 4m^2_{\mathrm{\tau}}}{4}} \approx \SI{1.76}{GeV}
\end{align}

\subsubsection{6.3.1 ptw}
Aufgrund der Impulserhaltung kann der Transversalimpuls des W-Bosons \texttt{ptw} aus den Messungen des transversalen Elektronimpulses und dem fehlenden transversalen Impuls (aufgrund des Neutrinos) bestimmt werden:
\begin{align}
	p_\mathrm{T,W}=\sqrt{(p_\mathrm{e,x} + p_\mathrm{mis, x})^2 + (p_\mathrm{e,y} + p_\mathrm{mis, y})^2}
\end{align}

\subsubsection{6.3.1 Herleitung}
Mit der Kettenregel folgt:
\begin{align}
	\frac{\mathrm{d}\sigma}{\mathrm{d}p_\mathrm{T}} = \frac{\mathrm{d}\sigma}{\mathrm{d}\cos\vartheta^*}\left|\frac{\mathrm{d}\cos\vartheta^*}{\mathrm{d}p_\mathrm{T}}\right| = \frac{\mathrm{d}\sigma}{\mathrm{d}\cos\vartheta^*}\frac{1}{\left|\frac{\mathrm{d}p_\mathrm{T}}{\mathrm{d}\cos\vartheta^*}\right|}
\end{align}
Es gilt $p_\mathrm{T}=p^*\cos\vartheta^*$ mit $p^*$ dem Impuls im Schwerpunktsystem, wobei für eine vernachlässigbar kleine Leptonmasse $p^*\approx\frac{1}{2}M_\mathrm{W}$ angenommen werden kann.
Dann gilt:
\begin{align}
	\frac{\mathrm{d}p_\mathrm{T}}{\mathrm{d}\cos\vartheta^*}=p^*\frac{\mathrm{d}\sin\vartheta^*}{\mathrm{d}\cos\vartheta^*}=-p^*\frac{\cos\vartheta^*}{\sin\vartheta^*}
\end{align}
Mit der Ableitung folgt für $\frac{\mathrm{d}\sigma}{\mathrm{d}p_\mathrm{T}}$:
\begin{align}
	\frac{\mathrm{d}\sigma}{\mathrm{d}p_\mathrm{T}}&=\frac{\mathrm{d}\sigma}{\mathrm{d}\cos\vartheta^*} \frac{1}{p^*\left|\frac{\cos\vartheta^*}{\sin\vartheta^*}\right|}\\
	&=\frac{\mathrm{d}\sigma}{\mathrm{d}\cos\vartheta^*} \frac{1}{p^*\left|\frac{p^*\sqrt{1-\sin^2\vartheta^*}}{p^*\sin\vartheta^*}\right|}\\
	&=\frac{\mathrm{d}\sigma}{\mathrm{d}\cos\vartheta^*} \frac{1}{p^*\left|\frac{\sqrt{{p^*}^2-{p^*}^2\sin^2\vartheta^*}}{p^*\sin\vartheta^*}\right|}\\
	&=\frac{\mathrm{d}\sigma}{\mathrm{d}\cos\vartheta^*} \frac{1}{p^*\frac{\sqrt{{p^*}^2-p_\mathrm{T}^2}}{p_\mathrm{T}}}\\
	&=\frac{\mathrm{d}\sigma}{\mathrm{d}\cos\vartheta^*} \frac{p_\mathrm{T}}{p^*\sqrt{{p^*}^2-p_\mathrm{T}^2}}\\
	&\approx \frac{\mathrm{d}\sigma}{\mathrm{d}\cos\vartheta^*} \frac{2p_\mathrm{T}}{M_\mathrm{W}} \frac{1}{\sqrt{\frac{1}{4}M_\mathrm{W}^2-p_\mathrm{T}^2}}
\end{align}
\subsubsection{6.4.1 1.)}
Die invariante Vier-Lepton-Masse muss mindestens so groß sein wie $2m_\mathrm{Z^0}$, wenn die Z-Bosonen reell sind.
Der Minimalfall tritt genau dann ein, wenn die Z-Bosonen ruhen und zusätzlichen Impuls haben.
Unterhalb dieser Schwelle können dennoch Zerfälle stattfinden, bei denen mindestens eines der Z-Bosonen virtuell auftritt.

\subsubsection{6.1.4 2.)}
Man erwartet eine Breit-Wigner-Verteilung der invarianten Masse mit Maximum bei der Masse des Higgs-Bosons.

\subsubsection{6.1.4 3.)}
In einem idealen Detektor würden sowohl Elektronen als auch Myonen sowie Jets vollständig detektiert werden, das heißt es findet eine ideale Energie- und Impulsmessung statt, so dass keine fehlende Energie auftritt.
Aufgrund der endlichen Auflösung eines realen Detektors tritt in diesem Fall eine geringe, aber von null verschiedene fehlende Energie auf.

\subsubsection{6.1.4 4.)}
Die im top-Zerfall auftretenden bottom Quarks hadronisieren zu Mesonen, die über die schwache Wechselwirkung zerfallen, wobei zusätzliche Leptonen im Endzustand erzeugt werden.

\subsubsection{6.1.4 5.)}
Der statistische Fehler auf die Anzahl Einträge in einem Bin ist $\sqrt{(N)}$ ($N$: Anzahl Einträge), in dem vorliegenden Fall also \num{10}.
Die Wahrscheinlichkeit, in einem Bin \num{130} Einträge zu finden entspricht damit einer Abweichung von $3\sigma$ vom Mittelwert, die durch eine Integration über die Wahrscheinlichkeitsdichte zu \SI{0.27}{\percent}$\approx\frac{1}{370}$ bestimmt werden kann.
Da dies für alle Bins gilt, ist die Wahrscheinlichkeit, in einem der \num{200} Bins eine Abweichung von $3\sigma$ festzustellen, $\frac{1}{370}\cdot200=\num{0.54}$.
Insgesamt lässt sich die Wahrscheinlichkeit $P$, in mindestens einem Bin eine Abweichung von $3\sigma$ zu finden, dann bestimmen aus der Wahrscheinlichkeit, in keinem Bin eine Abweichung zu finden:
\begin{align}
	P=1-\left(1-\frac{1}{370}\right)^{200}\approx \SI{42}{\percent}
\end{align}

\FloatBarrier
\subsection{Section in Appendix}
Appendix
\FloatBarrier

\end{appendix}

\end{document}
