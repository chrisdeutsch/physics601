% !TeX spellcheck = en_US
\documentclass[11pt,xcolor=dvipsnames,professionalfonts]{beamer}

% Pakete
\usepackage[utf8]{inputenc}
\usepackage[english]{babel}

% AMS Pakete
\usepackage{amsmath}
\usepackage{amsfonts}
\usepackage{amssymb}
\usepackage{mathtools}

% Text tools
\usepackage{textcomp}

% Einheiten
\usepackage{siunitx}
\sisetup{
	separate-uncertainty
}

% Grafiken
\usepackage{graphicx}
\usepackage{booktabs}
\usepackage{multirow}
\setbeamerfont{caption}{size=\footnotesize}
\setbeamertemplate{caption}{\raggedright\insertcaption\par}
\usepackage[percent]{overpic}

% Theme
\usetheme{Boadilla}
\usecolortheme{rose}
\useoutertheme{infolines}
\useinnertheme{rectangles}
\setbeamertemplate{itemize subitem}[triangle]

\usefonttheme[onlymath]{serif}

% [num] Zitationen
\setbeamertemplate{bibliography item}[text]

% Navigationsleiste ausschalten
\beamertemplatenavigationsymbolsempty

\author[Christian Bespin \& Christopher Deutsch]
{Christian Bespin \& Christopher Deutsch}

\title
{Analysis of $\mathrm{Z}^0$ Decays}

\subtitle
{}
%\logo{}

\institute[]
{Advanced Laboratory Course\\ Summer Term 16}

\date{May 30, 2016}

%\setbeamercovered{transparent}
%\setbeamertemplate{navigation symbols}{}

\newcommand{\beginbackup}{
	\newcounter{framenumbervorappendix}
	\setcounter{framenumbervorappendix}{\value{framenumber}}
}
\newcommand{\backupend}{
	\addtocounter{framenumbervorappendix}{-\value{framenumber}}
	\addtocounter{framenumber}{\value{framenumbervorappendix}} 
}

\begin{document}
\maketitle


\begin{frame}{Outline}
	\tableofcontents
\end{frame}

\section{Introduction}
\begin{frame}{Christian's Folien}
	Hier deine Folien
\end{frame}

\section{Theoretical Background}

\section{Part I: Analysis of Event Displays}

\section{Part II: Statistical Analysis of $\mathrm{Z}^0$ Decays}
\begin{frame}{Part II: Statistical Analysis of $\mathrm{Z}^0$ Decays}
	\begin{itemize}
		\setlength\itemsep{2.em}
		\item Analysis of data gathered by OPAL at LEP
		\begin{itemize}
			\setlength\itemsep{0.5em}
			\item event displays infeasible ($\sim \num{400000}$ events)
			\item data analysis software \texttt{PAW} used instead
		\end{itemize}
		
		\item Determination of
		\begin{itemize}
			\setlength\itemsep{0.5em}
			\item cross sections $\sigma_f$
			
			\item electroweak mixing $\sin^2\theta_\mathrm{W}$ (forward-backward asymmetry)
			
			\item mass $M_\mathrm{Z}$ and total decay-width $\Gamma_\mathrm{Z}$
			
			\item partial decay-widths $\Gamma_f$ (lepton universality, number of light neutrino generations $N_\nu$)
		\end{itemize}
	\end{itemize}
\end{frame}

\begin{frame}{Part II: Statistical Analysis of $\mathrm{Z}^0$ Decays}
	\textbf{Procedure:}
	\begin{enumerate}
		\setlength\itemsep{2.em}
		\item Analysis of Monte-Carlo data
		\begin{itemize}
			\setlength\itemsep{0.5em}
			\item refine selection cuts for the different decay-channels
			
			\item separate $s$- and $t$-channel in $\mathrm{e}^+ + \mathrm{e}^- \rightarrow \mathrm{e}^+ + \mathrm{e}^-$
			
			\item determine the efficiency of the applied cuts
		\end{itemize}
		
		\item Analysis of OPAL data
		\begin{itemize}
			\setlength\itemsep{0.5em}
			\item separate the decay-channels with the selection cuts
			
			\item efficiency formalism to obtain the total number of events
			
			\item calculation of the physical quantities
		\end{itemize}
	\end{enumerate}
\end{frame}

\subsection{Analysis of Monte-Carlo Data}

\begin{frame}{Analysis of Monte-Carlo Data}
	\begin{itemize}
		\setlength\itemsep{2.em}
		\item Simulated events with detector response separated into decay channels $\mathrm{e}, \mu, \tau, \mathrm{q}$ (\num{100000} events each)
		
		\item Available variables for each event:
		\begin{description}
			\item[\texttt{ncharged}] number of charged particles
			\item[\texttt{pcharged}] total momentum of charged particles
			\item[\texttt{e\_ecal}] total energy deposited in em.\ calorimeter
			\item[\texttt{e\_hcal}] total energy deposited in had.\ calorimeter
			\item[\texttt{cos\_thet}] angle between inc.\ and outg.\ positively charged particle
		\end{description}
	\end{itemize}
\end{frame}

\begin{frame}{Acceptance of the OPAL-Detector}
	\begin{columns}
		\column[c]{0.66\textwidth}
			\begin{overpic}[height=0.9\textheight, trim=0 0 0 20, clip]{./data/tag2/uncut/cropped/pcharged_uncut.pdf}
				\put(31.3,86.7){\fcolorbox{black}{white}{\makebox(12,3)[c]{\footnotesize electrons}}}
				\put(78.8,86.7){\fcolorbox{black}{white}{\makebox(12,3)[c]{\footnotesize muons}}}
				\put(31.3,39.3){\fcolorbox{black}{white}{\makebox(12,3)[c]{\footnotesize taus}}}
				\put(78.8,39.3){\fcolorbox{black}{white}{\makebox(12,3)[c]{\footnotesize hadrons}}}
			\end{overpic}
		
		\column[c]{0.33\textwidth}
			\begin{itemize}
				\item $\left| \cos\theta \right| < 0.9$
			\end{itemize}
	\end{columns}
\end{frame}

\begin{frame}{Refining the Cuts}
	\begin{center}
		\begin{tabular}{lcccc}
	\toprule
	 & $e$ & $\mu$ & $\tau$ & hadronic \\
	 \midrule
	\texttt{ncharged} & $<5$  & $<5$  & $\leq 5$ & $>7$ \\
	\texttt{pcharged} & $>30$ & $>65$ & $<65$ & --\\
	\texttt{e\_ecal}  & $>60$ & $<8$  & $<60$ & -- \\
	\texttt{e\_hcal}  & $<2$  & $<10$ & -- & -- \\
	\bottomrule
\end{tabular}
	\end{center}
\end{frame}

\begin{frame}{Separating $s$- and $t$-Channel in $\mathrm{e}^+ + \mathrm{e}^- \rightarrow \mathrm{e}^+ + \mathrm{e}^-$}
\end{frame}


\begin{frame}{Efficiency Formalism}
	\begin{center}
		\begin{tabular}{rS[table-format=6.0]S[table-format=6.0]S[table-format=6.0]S[table-format=6.0]}
	\toprule
	&\multicolumn{4}{c}{selected channel}  \\ \cmidrule(r){2-5}
	& {electrons} & {muons} & {tauons} & {hadrons}\\
	\midrule
	total & 100000 & 100000 & 100000 & 100000\\
	precut & 93454 & 93979 & 79051 & 97848\\
	$n_\mathrm{e}$ & 20229 & 0 & 692 & 6 \\
	$n_\mathrm{\mu}$ & 1 & 78537 & 4089 & 0 \\
	$n_\mathrm{\tau}$ & 117 & 549 & 76651 & 503 \\
	$n_\mathrm{q}$ & 0 & 0 & 934 & 97533 \\
 	\bottomrule
\end{tabular}
	\end{center}
	
	Definition of the efficiency matrix:
	\begin{align*}
		\vec{n} = E \cdot \vec{N} \qquad E_{ij} = \frac{n_{ij}}{N_j}
	\end{align*}
\end{frame}


\begin{frame}{Efficiency Formalism}
	Efficiency matrix for our cuts:
	\begin{align*}
	\footnotesize
	E = \begin{pmatrix}
	0.59259 & 0.00001 & 0.00117 & 0.00000 \\
	0.00000 & 0.78537 & 0.00549 & 0.00000 \\
	0.02027 & 0.04089 & 0.76651 & 0.00934 \\
	0.00018 & 0.00000 & 0.00503 & 0.97533 \\
	\end{pmatrix}
	\end{align*}
	
	With errors from binomial statistics:
	\begin{align*}
	\footnotesize
	\Delta E = \begin{pmatrix}
	0.00266 & 0.00000 & 0.00000 & 0.00000 \\
	0.00000 & 0.00130 & 0.00024 & 0.00000 \\
	0.00077 & 0.00063 & 0.00134 & 0.00031 \\
	0.00008 & 0.00000 & 0.00023 & 0.00050 \\
	\end{pmatrix}
	\end{align*}
		
	\textbf{Problem:} Experimentally only $\vec{n}$ is measured. Invert the efficiency matrix.
	\begin{align*}
		\vec{N} = E^{-1} \cdot \vec{n}
	\end{align*}
\end{frame}

\begin{frame}{Efficiency Formalism}
	Numerical inversion:
	\begin{align*}
	\footnotesize
	E^{-1} = \begin{pmatrix}
	1.68758 & 0.00011 &-0.00258 & 0.00002 \\
	0.00312 & 1.27376 &-0.00912 & 0.00009 \\
	-0.04465 &-0.06796 & 1.30525 &-0.01250 \\
	-0.00007 & 0.00035 &-0.00673 & 1.02536 \\
	\end{pmatrix}
	\end{align*}
	
	Errors by Monte-Carlo error propagation (\num{100000} samples):
	\begin{align*}
	\footnotesize
	\Delta E^{-1} = \begin{pmatrix}
	0.00756 & 0.00001 & 0.00002 & 0.00001 \\
	0.00002 & 0.00211 & 0.00039 & 0.00001 \\
	0.00170 & 0.00105 & 0.00227 & 0.00041 \\
	0.00013 & 0.00002 & 0.00030 & 0.00052 \\
	\end{pmatrix}
	\end{align*}
	
	\textbf{Now:} Apply cuts to experimental data to get $\vec{n}$ and calculate total events $\vec{N}$.
\end{frame}

\subsection{Analysis of OPAL Data}

\begin{frame}{Analysis of OPAL Data}
\end{frame}


\section{Conclusion}

\begin{frame}
	\begin{center}
		\LARGE
		\textbf{Thank you for your attention!}
	\end{center}
\end{frame}

\begin{frame}{Bibliography}
	\scriptsize
	\begin{thebibliography}{9}
		\bibitem[PDG]{pdg}
			K.A. Olive \textit{et al.} (Particle Data Group),
			\emph{The Review of Particle Physics},
			Chin.\ Phys.\ C, \textbf{38}, 090001 (2014) and 2015 update.
	\end{thebibliography}
\end{frame}

\beginbackup

\begin{frame}{Backup Slides}
	Backup Slides
\end{frame}

\backupend

\end{document}