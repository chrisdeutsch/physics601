% !TeX spellcheck = en_US
\documentclass[11pt,xcolor=dvipsnames,professionalfonts]{beamer}

% Pakete
\usepackage[utf8]{inputenc}
\usepackage[english]{babel}

% AMS Pakete
\usepackage{amsmath}
\usepackage{amsfonts}
\usepackage{amssymb}
\usepackage{mathtools}

% Text tools
\usepackage{textcomp}

% Einheiten
\usepackage{siunitx}
\sisetup{
	separate-uncertainty
}

% Grafiken
\usepackage{graphicx}
\usepackage{booktabs}
\usepackage{multirow}
\setbeamerfont{caption}{size=\footnotesize}
\setbeamertemplate{caption}{\raggedright\insertcaption\par}
\usepackage[percent]{overpic}

% Theme
\usetheme{Boadilla}
\usecolortheme{rose}
\useoutertheme{infolines}
\useinnertheme{rectangles}
\setbeamertemplate{itemize subitem}[triangle]

\usefonttheme[onlymath]{serif}

% [num] Zitationen
\setbeamertemplate{bibliography item}[text]

% Navigationsleiste ausschalten
\beamertemplatenavigationsymbolsempty

\author[Christian Bespin \& Christopher Deutsch]
{Christian Bespin \& Christopher Deutsch}

\title
{Analysis of $\mathrm{Z}^0$ Decays}

\subtitle
{}
%\logo{}

\institute[]
{Advanced Laboratory Course\\ Summer Term 16}

\date{May 30, 2016}

%\setbeamercovered{transparent}
%\setbeamertemplate{navigation symbols}{}

\newcommand{\beginbackup}{
	\newcounter{framenumbervorappendix}
	\setcounter{framenumbervorappendix}{\value{framenumber}}
}
\newcommand{\backupend}{
	\addtocounter{framenumbervorappendix}{-\value{framenumber}}
	\addtocounter{framenumber}{\value{framenumbervorappendix}} 
}

\begin{document}
\maketitle


\begin{frame}{Outline}
	\tableofcontents
\end{frame}

\section{Introduction}
\begin{frame}{Christian's Folien}
	Hier deine Folien
\end{frame}

\section{Theoretical Background}

\section{Part I: Analysis of Event Displays}

\section{Part II: Statistical Analysis of $\mathrm{Z}^0$ Decays}
\begin{frame}{Part II: Statistical Analysis of $\mathrm{Z}^0$ Decays}
	\begin{itemize}
		\setlength\itemsep{2.em}
		\item Analysis of data gathered by OPAL at LEP
		\begin{itemize}
			\setlength\itemsep{0.5em}
			\item event displays infeasible ($\sim \num{400000}$ events)
			\item data analysis software \texttt{PAW} used instead
		\end{itemize}
		
		\item Determination of
		\begin{itemize}
			\setlength\itemsep{0.5em}
			\item cross sections $\sigma_f$
			
			\item mass $M_\mathrm{Z}$ and total decay-width $\Gamma_\mathrm{Z}$
			
			\item partial decay-widths $\Gamma_f$ (lepton universality, number of light neutrino generations $N_\nu$)
			
			\item electroweak mixing $\sin^2\theta_\mathrm{W}$ (forward-backward asymmetry)
		\end{itemize}
	\end{itemize}
\end{frame}

\begin{frame}{Part II: Statistical Analysis of $\mathrm{Z}^0$ Decays}
	\textbf{Procedure:}
	\begin{enumerate}
		\setlength\itemsep{2.em}
		\item Analysis of Monte-Carlo data
		\begin{itemize}
			\setlength\itemsep{0.5em}
			\item refine selection cuts for the different decay-channels
			
			\item separate $s$- and $t$-channel in $\mathrm{e}^+ + \mathrm{e}^- \rightarrow \mathrm{e}^+ + \mathrm{e}^-$
			
			\item determine the efficiency of the applied cuts
		\end{itemize}
		
		\item Analysis of OPAL data
		\begin{itemize}
			\setlength\itemsep{0.5em}
			\item separate the decay-channels with the selection cuts
			
			\item efficiency formalism to obtain the total number of events
			
			\item calculation of the physical quantities
		\end{itemize}
	\end{enumerate}
\end{frame}

\subsection{Analysis of Monte-Carlo Data}

\begin{frame}{Analysis of Monte-Carlo Data}
	\begin{itemize}
		\setlength\itemsep{2.em}
		\item Simulated events with detector response separated into decay channels $\mathrm{e}, \mu, \tau, \mathrm{q}$ (\num{100000} events each)
		
		\item Available variables for each event:
		\begin{description}
			\item[\texttt{ncharged}] number of charged particles
			\item[\texttt{pcharged}] total momentum of charged particles
			\item[\texttt{e\_ecal}] total energy deposited in em.\ calorimeter
			\item[\texttt{e\_hcal}] total energy deposited in had.\ calorimeter
			\item[\texttt{cos\_thet}] angle between inc.\ and outg.\ positively charged particle
			\item[\texttt{e\_lep}] energy of the beam at LEP
		\end{description}
	\end{itemize}
\end{frame}

\begin{frame}{Refining the Cuts -- \texttt{pcharged}}
	\begin{center}
		\begin{overpic}[height=0.9\textheight, trim=0 0 0 20, clip]{./talkfigs/pdf/pcharged_uncut.pdf}
			\put(31.3,86.7){\fcolorbox{black}{white}{\makebox(12,3)[c]{\footnotesize electrons}}}
			\put(78.8,86.7){\fcolorbox{black}{white}{\makebox(12,3)[c]{\footnotesize muons}}}
			\put(31.3,39.3){\fcolorbox{black}{white}{\makebox(12,3)[c]{\footnotesize taus}}}
			\put(78.8,39.3){\fcolorbox{black}{white}{\makebox(12,3)[c]{\footnotesize hadrons}}}
		\end{overpic}
	\end{center}
\end{frame}

\begin{frame}[noframenumbering]{Refining the Cuts -- \texttt{pcharged}}
	\begin{center}
		\begin{overpic}[height=0.9\textheight, trim=0 0 0 20, clip]{./talkfigs/pdf/pcharged_coscut.pdf}
			\put(31.3,86.7){\fcolorbox{black}{white}{\makebox(12,3)[c]{\footnotesize electrons}}}
			\put(78.8,86.7){\fcolorbox{black}{white}{\makebox(12,3)[c]{\footnotesize muons}}}
			\put(31.3,39.3){\fcolorbox{black}{white}{\makebox(12,3)[c]{\footnotesize taus}}}
			\put(78.8,39.3){\fcolorbox{black}{white}{\makebox(12,3)[c]{\footnotesize hadrons}}}
		\end{overpic}
	\end{center}
\end{frame}

\begin{frame}<1>[label=cutsummary]{Refining the Cuts -- Summary}
	\begin{itemize}
		\item<1-> electrons and muons: $\left| \cos\theta \right| < 0.9$ (unreconstructible events)
		\item<6-> electrons: $\cos\theta < 0.5$ ($s$-channel)
	\end{itemize}
	\vfill
	\pause
	\begin{center}
		\begin{tabular}{lcccc}
			\toprule
			&\multicolumn{4}{c}{selected channel}  \\ \cmidrule(r){2-5}
			& electrons & muons & tauons & hadrons \\
			\midrule
			\texttt{pcharged} & \only<2->{$> 30$} & \only<2->{$> 55$} & \only<2->{$>1\,\land\,<75$} & \only<2->{--}\\
			\texttt{ncharged} & \only<3->{$< 5$}  & \only<3->{$< 4$}  & \only<3->{$< 8$} & \only<3->{$\geq 8$} \\
			\texttt{e\_ecal}  & \only<4->{$> 70$} & \only<4->{$< 15$}  & \only<4->{$< 75$} & \only<4->{$> 20$} \\
			\texttt{e\_hcal}  & \only<5->{$< 15$}  & \only<5->{$< 25$} & \only<5->{--} & \only<5->{--} \\
			\bottomrule
		\end{tabular}
	\end{center}
\end{frame}

\begin{frame}{Refining the Cuts -- \texttt{pcharged}}
	\begin{center}
		\begin{overpic}[height=0.9\textheight, trim=0 0 0 20, clip]{./talkfigs/pdf/pcharged_uncut.pdf}
			\put(31.3,86.7){\fcolorbox{black}{white}{\makebox(12,3)[c]{\footnotesize electrons}}}
			\put(78.8,86.7){\fcolorbox{black}{white}{\makebox(12,3)[c]{\footnotesize muons}}}
			\put(31.3,39.3){\fcolorbox{black}{white}{\makebox(12,3)[c]{\footnotesize taus}}}
			\put(78.8,39.3){\fcolorbox{black}{white}{\makebox(12,3)[c]{\footnotesize hadrons}}}
		\end{overpic}
	\end{center}
\end{frame}

\begin{frame}[noframenumbering]{Refining the Cuts -- \texttt{pcharged}}
	\begin{center}
		\begin{overpic}[height=0.9\textheight, trim=0 0 0 20, clip]{./talkfigs/pdf/pcharged_cuts.pdf}
			\put(31.3,86.7){\fcolorbox{black}{white}{\makebox(12,3)[c]{\footnotesize electrons}}}
			\put(78.8,86.7){\fcolorbox{black}{white}{\makebox(12,3)[c]{\footnotesize muons}}}
			\put(31.3,39.3){\fcolorbox{black}{white}{\makebox(12,3)[c]{\footnotesize taus}}}
			\put(78.8,39.3){\fcolorbox{black}{white}{\makebox(12,3)[c]{\footnotesize hadrons}}}
		\end{overpic}
	\end{center}
\end{frame}

\againframe<2>{cutsummary}

\begin{frame}{Refining the Cuts -- \texttt{ncharged}}
	\begin{center}
		\begin{overpic}[height=0.9\textheight, trim=0 0 0 20, clip]{./talkfigs/pdf/ncharged_uncut.pdf}
			\put(31.3,86.7){\fcolorbox{black}{white}{\makebox(12,3)[c]{\footnotesize electrons}}}
			\put(78.8,86.7){\fcolorbox{black}{white}{\makebox(12,3)[c]{\footnotesize muons}}}
			\put(31.3,39.3){\fcolorbox{black}{white}{\makebox(12,3)[c]{\footnotesize taus}}}
			\put(78.8,39.3){\fcolorbox{black}{white}{\makebox(12,3)[c]{\footnotesize hadrons}}}
		\end{overpic}
	\end{center}
\end{frame}

\begin{frame}[noframenumbering]{Refining the Cuts -- \texttt{ncharged}}
	\begin{center}
		\begin{overpic}[height=0.9\textheight, trim=0 0 0 20, clip]{./talkfigs/pdf/ncharged_cuts.pdf}
			\put(31.3,86.7){\fcolorbox{black}{white}{\makebox(12,3)[c]{\footnotesize electrons}}}
			\put(78.8,86.7){\fcolorbox{black}{white}{\makebox(12,3)[c]{\footnotesize muons}}}
			\put(31.3,39.3){\fcolorbox{black}{white}{\makebox(12,3)[c]{\footnotesize taus}}}
			\put(78.8,39.3){\fcolorbox{black}{white}{\makebox(12,3)[c]{\footnotesize hadrons}}}
		\end{overpic}
	\end{center}
\end{frame}

\againframe<3>{cutsummary}

\begin{frame}{Refining the Cuts -- \texttt{e\_ecal}}
	\begin{center}
		\begin{overpic}[height=0.9\textheight, trim=0 0 0 20, clip]{./talkfigs/pdf/e_ecal_uncut.pdf}
			\put(31.3,86.7){\fcolorbox{black}{white}{\makebox(12,3)[c]{\footnotesize electrons}}}
			\put(78.8,86.7){\fcolorbox{black}{white}{\makebox(12,3)[c]{\footnotesize muons}}}
			\put(31.3,39.3){\fcolorbox{black}{white}{\makebox(12,3)[c]{\footnotesize taus}}}
			\put(78.8,39.3){\fcolorbox{black}{white}{\makebox(12,3)[c]{\footnotesize hadrons}}}
		\end{overpic}
	\end{center}
\end{frame}

\begin{frame}[noframenumbering]{Refining the Cuts -- \texttt{e\_ecal}}
	\begin{center}
		\begin{overpic}[height=0.9\textheight, trim=0 0 0 20, clip]{./talkfigs/pdf/e_ecal_cuts.pdf}
			\put(31.3,86.7){\fcolorbox{black}{white}{\makebox(12,3)[c]{\footnotesize electrons}}}
			\put(78.8,86.7){\fcolorbox{black}{white}{\makebox(12,3)[c]{\footnotesize muons}}}
			\put(31.3,39.3){\fcolorbox{black}{white}{\makebox(12,3)[c]{\footnotesize taus}}}
			\put(78.8,39.3){\fcolorbox{black}{white}{\makebox(12,3)[c]{\footnotesize hadrons}}}
		\end{overpic}
	\end{center}
\end{frame}

\againframe<4>{cutsummary}

\begin{frame}{Refining the Cuts -- \texttt{e\_hcal}}
	\begin{center}
		\begin{overpic}[height=0.9\textheight, trim=0 0 0 20, clip]{./talkfigs/pdf/e_hcal_uncut.pdf}
			\put(31.3,86.7){\fcolorbox{black}{white}{\makebox(12,3)[c]{\footnotesize electrons}}}
			\put(78.8,86.7){\fcolorbox{black}{white}{\makebox(12,3)[c]{\footnotesize muons}}}
			\put(31.3,39.3){\fcolorbox{black}{white}{\makebox(12,3)[c]{\footnotesize taus}}}
			\put(78.8,39.3){\fcolorbox{black}{white}{\makebox(12,3)[c]{\footnotesize hadrons}}}
		\end{overpic}
	\end{center}
\end{frame}

\begin{frame}[noframenumbering]{Refining the Cuts -- \texttt{e\_hcal}}
	\begin{center}
		\begin{overpic}[height=0.9\textheight, trim=0 0 0 20, clip]{./talkfigs/pdf/e_hcal_cuts.pdf}
			\put(31.3,86.7){\fcolorbox{black}{white}{\makebox(12,3)[c]{\footnotesize electrons}}}
			\put(78.8,86.7){\fcolorbox{black}{white}{\makebox(12,3)[c]{\footnotesize muons}}}
			\put(31.3,39.3){\fcolorbox{black}{white}{\makebox(12,3)[c]{\footnotesize taus}}}
			\put(78.8,39.3){\fcolorbox{black}{white}{\makebox(12,3)[c]{\footnotesize hadrons}}}
		\end{overpic}
	\end{center}
\end{frame}

\againframe<5>{cutsummary}

\begin{frame}{Separating $s$- and $t$-Channel in $\mathrm{e}^+ + \mathrm{e}^- \rightarrow \mathrm{e}^+ + \mathrm{e}^-$}
	\begin{columns}
		\column[c]{0.5\textwidth}
			\begin{itemize}
				\setlength\itemsep{2.em}
				\item<2-> $t$-channel dominant at small scattering angles
				
				\item<3-> remove most $t$-channel events by cutting $\cos\theta < 0.5$
				
				\item<4-> correction for the number of $s$-channel events
				\begin{align*}
					\kappa &= \frac{\int_{-1}^{1} (1 + \cos^2\theta) \, \mathrm{d}\cos\theta}{\int_{-0.9}^{0.5} (1 + \cos^2\theta) \, \mathrm{d}\cos\theta} \\
					&\approx  1.5829
				\end{align*}
			\end{itemize}
		\column[c]{0.5\textwidth}
			\includegraphics<1>[width=1.0\textwidth]{./talkfigs/pdf/cos_thet_uncut.pdf}
			\includegraphics<2>[width=1.0\textwidth]{./talkfigs/pdf/cos_thet_annotated.pdf}
			\includegraphics<3->[width=1.0\textwidth]{./talkfigs/pdf/cos_thet_cuts.pdf}
	\end{columns}
\end{frame}

\againframe<6>{cutsummary}

\begin{frame}{Efficiency Matrix}
	Using the chosen cuts quantify the:
	\begin{itemize}
		\setlength\itemsep{1.em}
		\item fraction of correctly identified events
		\item probability of misidentifying a event
	\end{itemize}
	\vspace{0.7cm}
	Matrix formalism:
	\begin{itemize}
		\setlength\itemsep{1.em}
		\item number of events in different channels separated in vectors
		\begin{align*}
			&\vec{n} = \begin{pmatrix}
			\alert{n_\mathrm{e,s}} & n_\mathrm{\mu} & n_\mathrm{\tau} & n_\mathrm{q}
			\end{pmatrix}^\mathrm{T}
			\qquad
			&\vec{N} = \begin{pmatrix}
			\alert{N_\mathrm{e,s}} & N_\mathrm{\mu} & N_\mathrm{\tau} & N_\mathrm{q}
			\end{pmatrix}^\mathrm{T}
		\end{align*}
		$\vec{n}$: number of events after cuts\qquad $\vec{N}$: total number of events 
		
		\item efficiency matrix definition
		\begin{align*}
			\vec{n} = E \cdot \vec{N}
		\end{align*}
		
		
	\end{itemize}
\end{frame}


\begin{frame}{Efficiency Matrix}
	Apply cuts to Monte-Carlo datasets:
	\begin{center}
		\begin{tabular}{rS[table-format=6.0]S[table-format=6.0]S[table-format=6.0]S[table-format=6.0]}
	\toprule
	&\multicolumn{4}{c}{dataset}  \\ \cmidrule(r){2-5}
	$j$ & {$\mathrm{e}$} & {$\mathrm{\mu}$} & {$\mathrm{\tau}$} & {$\mathrm{q}$}\\
	\midrule
	total & 100000 & 100000 & 100000 & 100000\\
	precut & 93454 & 93979 & 79051 & 97848\\
	$n_{\mathrm{e}j}$ & 20229 & 0 & 692 & 6 \\
	$n_{\mathrm{\mu}j}$ & 1 & 78537 & 4089 & 0 \\
	$n_{\mathrm{\tau}j}$ & 117 & 549 & 76651 & 503 \\
	$n_{\mathrm{q}j}$ & 0 & 0 & 934 & 97533 \\
 	\bottomrule
\end{tabular}
	\end{center}
	
	Calculate matrix elements:
	\begin{align*}
		E_{ij} = \frac{n_{ij}}{N_j} \qquad\qquad N_\mathrm{e,s} = n_\mathrm{ee} \cdot \underbrace{\frac{100000}{93454}}_{\substack{\text{precut}\\\text{efficiency}}} \cdot \underbrace{\kappa}_{\substack{\text{correct.} \\ \text{due to} \\ \text{ang.\ cut}}} = 34136
	\end{align*}
\end{frame}


\begin{frame}{Efficiency Matrix}
	Efficiency matrix for our cuts:
	\begin{align*}
	\footnotesize
	E = \begin{pmatrix}
	0.59259 & 0.00001 & 0.00117 & 0.00000 \\
	0.00000 & 0.78537 & 0.00549 & 0.00000 \\
	0.02027 & 0.04089 & 0.76651 & 0.00934 \\
	0.00018 & 0.00000 & 0.00503 & 0.97533 \\
	\end{pmatrix}
	\end{align*}
	
	With errors from binomial statistics:
	\begin{align*}
	\footnotesize
	\Delta E = \begin{pmatrix}
	0.00266 & 0.00000 & 0.00000 & 0.00000 \\
	0.00000 & 0.00130 & 0.00024 & 0.00000 \\
	0.00077 & 0.00063 & 0.00134 & 0.00031 \\
	0.00008 & 0.00000 & 0.00023 & 0.00050 \\
	\end{pmatrix}
	\end{align*}
		
	\textbf{Problem:} Experimentally only $\vec{n}$ is measured. Invert the efficiency matrix.
	\begin{align*}
		\vec{N} = E^{-1} \cdot \vec{n}
	\end{align*}
\end{frame}

\begin{frame}{Efficiency Matrix}
	Numerical inversion:
	\begin{align*}
	\footnotesize
	E^{-1} = \begin{pmatrix}
	1.68758 & 0.00011 &-0.00258 & 0.00002 \\
	0.00312 & 1.27376 &-0.00912 & 0.00009 \\
	-0.04465 &-0.06796 & 1.30525 &-0.01250 \\
	-0.00007 & 0.00035 &-0.00673 & 1.02536 \\
	\end{pmatrix}
	\end{align*}
	
	Errors by Monte-Carlo error propagation (\num{100000} samples):
	\begin{align*}
	\footnotesize
	\Delta E^{-1} = \begin{pmatrix}
	0.00756 & 0.00001 & 0.00002 & 0.00001 \\
	0.00002 & 0.00211 & 0.00039 & 0.00001 \\
	0.00170 & 0.00105 & 0.00227 & 0.00041 \\
	0.00013 & 0.00002 & 0.00030 & 0.00052 \\
	\end{pmatrix}
	\end{align*}
\end{frame}

\subsection{Analysis of OPAL Data}

\begin{frame}{Analysis of OPAL Data}
	\begin{itemize}
		\setlength\itemsep{1.5em}
		\item total number of events $\vec{N}$ now unknown
		\begin{itemize}
			\setlength\itemsep{.5em}
			\item apply cuts to experimental data to get $\vec{n}$
			\item determine $\vec{N}$ using:\quad $\vec{N} = E^{-1} \cdot \vec{n}$
		\end{itemize}
		
		\item determine cross section $\sigma_i$ as a function of cm.\ energy
		\begin{itemize}
			\setlength\itemsep{.5em}
			\item data contains events at multiple \texttt{e\_lep} -- select energies using cuts
			\item after identifying the channels calculate cross section
			\begin{align*}
				\sigma_i = \frac{N_i}{\int \mathcal{L} \, \mathrm{d}t} + \underbrace{\mathrm{corr}_i(\sqrt{s})}_{\substack{\text{radiation} \\ \text{correction}}} \qquad i \in \left\{ \mathrm{e}, \mathrm{\mu}, \mathrm{\tau}, \mathrm{q} \right\}
			\end{align*}
			with integrated luminosity $\int \mathcal{L} \, \mathrm{d}t$ at a given energy
		\end{itemize}
	\end{itemize}
\end{frame}

\begin{frame}{Cross Section \& Breit-Wigner Fit}
	\begin{itemize}
		\setlength\itemsep{1.5em}
		
		\item for each channel fit a Breit-Wigner resonance
			\begin{align*}
			\sigma(E_\mathrm{cm}; A, M_\mathrm{Z}, \Gamma_\mathrm{Z}) = \frac{A}{2\pi} \frac{\Gamma_\mathrm{Z}}{\left( E_\mathrm{cm} - M_\mathrm{Z} \right)^2 + \Gamma_\mathrm{Z}^2 / 4}
			\end{align*}
			
			\begin{itemize}
				\setlength\itemsep{.5em}
				\item mass $M_\mathrm{Z}$
				\item total decay width $\Gamma_\mathrm{Z}$
				\item factor $A$ proportional to the cross section at peak
				\begin{align*}
					\sigma_\mathrm{peak} = \frac{2A}{\pi \Gamma_\mathrm{Z}}
				\end{align*}
			\end{itemize}
		
	\end{itemize}
\end{frame}

\begin{frame}{Cross Section \& Breit-Wigner Fit}
	\centering
	\includegraphics[height=0.9\textheight]{./talkfigs/pdf/cross_sections.pdf}
\end{frame}

\begin{frame}{Cross Section \& Breit-Wigner Fit}
	\begin{center}
		\begin{tabular}{lS[table-format=2.2,table-figures-uncertainty=3]S[table-format=1.2,table-figures-uncertainty=3]S[table-format=2.3,table-figures-uncertainty=4]S[table-format=1.3]}
	\toprule
	{}& {$M_\mathrm{Z}$ / GeV} & {$\Gamma_\mathrm{Z}$ / GeV} & {$\sigma^\mathrm{peak}$ / nb} & {$\chi_\mathrm{red}^2$} \\
	\midrule
	electrons & 90.99 +- 0.24 & 2.47 +- 0.11 & 1.934 +- 0.077 & 2.796 \\
	muons & 91.18 +- 0.12 & 2.47 +- 0.05 & 1.966 +- 0.033 & 0.847 \\
	tauons & 91.22 +- 0.26 & 2.72 +- 0.10 & 2.015 +- 0.069 & 2.451 \\
	hadrons & 91.19 +- 0.70 & 2.53 +- 0.02 & 41.138 +- 0.270 & 1.776 \\	
	\midrule
	mean & 91.15 +- 0.10 & 2.52 +- 0.02 & {--} & {--} \\
	\bottomrule	
\end{tabular}
	\end{center}
	\vfill
	Error weighted means:
	\begin{alignat*}{2}
		&M_\mathrm{Z} = \SI{91.15 +- 0.1}{GeV} \qquad &&M_\mathrm{Z}^\mathrm{PDG} = \SI{91.1876 +- 0.0021}{GeV}\\[0.5em]
		&\Gamma_\mathrm{Z} = \SI{2.52 +- 0.02}{GeV} \qquad &&\Gamma_\mathrm{Z}^\mathrm{PDG} = \SI{2.4952 +- 0.0023}{GeV}
	\end{alignat*}
\end{frame}

\begin{frame}{Partial Decay Widths}
	\begin{itemize}
		\item cross section at resonance from theory
			\begin{align*}
			\sigma_f^\mathrm{peak} = \frac{12 \pi}{M_\mathrm{Z}^2} \cdot \frac{\alert{\Gamma_\mathrm{e}}}{\Gamma_\mathrm{Z}} \cdot \frac{\Gamma_f}{\Gamma_\mathrm{Z}} \qquad f \in \left\{ \mathrm{e}, \mathrm{\mu}, \mathrm{\tau}, \mathrm{q} \right\}
			\end{align*}
		
		\begin{enumerate}
			\item calculate $\Gamma_\mathrm{e}$
				 \begin{align*}
				 	\Gamma_\mathrm{e} = \sqrt{\frac{M_\mathrm{Z}^2}{12 \pi} \, \Gamma_\mathrm{Z}^2 \sigma_\mathrm{e}^\mathrm{peak}}
				 \end{align*}
				 
			\item use $\Gamma_\mathrm{e}$ to determine $\Gamma_f$ for $f \in \left\{ \mathrm{\mu}, \mathrm{\tau}, \mathrm{q} \right\}$
				\begin{align*}
					\Gamma_f = \frac{M_\mathrm{Z}^2}{12 \pi } \, \frac{\Gamma_\mathrm{Z}^2 }{\Gamma_\mathrm{e}} \sigma_f^\mathrm{peak}
				\end{align*}
		\end{enumerate}
	\end{itemize}	
\end{frame}

\begin{frame}{Partial Decay Widths}
	\begin{itemize}
		\setlength\itemsep{2em}
		\item resulting partial decay widths
			\begin{alignat*}{2}
			&\Gamma_\mathrm{e} = \SI{83.6 +- 1.8}{MeV} \qquad
			&&\Gamma_\mathrm{\mu} = \SI{85.0 +- 2.3}{MeV}\\
			&\Gamma_\mathrm{\tau} = \SI{87.1 +- 3.5}{MeV}\qquad
			&&\Gamma_\mathrm{hadrons} = \SI{1778.5 +- 39}{MeV}
			\end{alignat*}
		
		\item literature values
		\begin{alignat*}{2}
		&\Gamma_\ell^\mathrm{PDG} = \SI{83.984 +- 0.086}{MeV} \qquad
		&&\Gamma_\mathrm{hadrons}^\mathrm{PDG} = \SI{1744.4 +- 2.0}{MeV}
		\end{alignat*}
		
		\item consistent with lepton universality
			\begin{align*}
				&\frac{\Gamma_\mathrm{\mu}}{\Gamma_\mathrm{e}} = \num{1.017 +- 0.044} \qquad 
				&\frac{\Gamma_\mathrm{\tau}}{\Gamma_\mathrm{e}} = \num{1.042 +- 0.055}
			\end{align*}
	\end{itemize}
\end{frame}

\begin{frame}{Number of Light Neutrino Generations}
	\begin{itemize}
		\setlength\itemsep{2em}
		\item full decay width of the Z-boson
		\begin{align*}
			\Gamma_\mathrm{Z} = \Gamma_\mathrm{e} + \Gamma_\mathrm{\mu} + \Gamma_\mathrm{\tau} + N_\nu \Gamma_\nu + \Gamma_\mathrm{hadrons}
		\end{align*}
		with $N_\nu$ number of neutrino generations $M_\nu < M_\mathrm{Z} / 2$
		
		\item $\Gamma_\nu$ known from theory: \quad $\Gamma_\nu = \SI{167.6}{MeV}$
		
		\item number of light neutrino generations
			\begin{align*}
			N_\nu &= \frac{\Gamma_\mathrm{Z} - \Gamma_\mathrm{e} - \Gamma_\mathrm{\mu} - \Gamma_\mathrm{\tau} - \Gamma_\mathrm{hadrons}}{\Gamma_\nu} \\[1.0em]
			&= \num{2.94 +- 0.24} \quad \text{(using measured values)}
			\end{align*}
	\end{itemize}
\end{frame}

\begin{frame}{Forward-Backward Asymmetry}
	\begin{itemize}
		\setlength\itemsep{1.5em}
		\item asymmetric number of events in forward- and backward-hemisphere
		\begin{align*}
			\cos\theta > 0~\text{(forward)} \qquad \cos\theta < 0~\text{(backward)}
		\end{align*}
		
		\item observable in all leptonic decay channels -- clearest signature for muons
		
		
		\item forward-backward asymmetry
			\begin{align*}
				A_\mathrm{FB} = \frac{n_\mathrm{F} - n_\mathrm{B}}{n_\mathrm{F} + n_\mathrm{B}} + \underbrace{\mathrm{corr}(\sqrt{s})}_{\substack{\text{radiation} \\ \text{correction}}}
			\end{align*}
			function of cm.\ energy
	\end{itemize}
\end{frame}

\begin{frame}{Forward-Backward Asymmetry}
	\begin{columns}
		\column[c]{0.45\textwidth}
		\begin{itemize}
			\setlength\itemsep{1.5em}
			\item asymmetry at peak of resonance
			\begin{align*}
				A_\mathrm{FB}^\mathrm{peak} \approx 3\left( 1 - 4 \sin^2\theta_\mathrm{W} \right)
			\end{align*}
			
			\item approx.\ peak asymmetry with measurement at $\sqrt{s} = \SI{91.24}{GeV}$
			\begin{align*}
			A_\mathrm{FB}^\mathrm{peak} \approx \num{0.0119 +- 0.0075}
			\end{align*}
		\end{itemize}
		\column[c]{0.55\textwidth}
			\begin{center}
				\includegraphics{./talkfigs/pdf/afb.pdf}
			\end{center}
	\end{columns}

\end{frame}

\begin{frame}{Forward-Backward Asymmetry}
		\begin{itemize}
			\setlength\itemsep{1.5em}
			\item calculate electroweak mixing
			\begin{align*}
			\sin^2\theta_\mathrm{W} &= \frac{1}{4} \left( 1 - \sqrt{A_\mathrm{FB}^\mathrm{peak} / 3} \right) \\[0.1em]
			&= \num{0.234 +- 0.005}
			\end{align*}
			
			\item good agreement with literature value
			\begin{align*}
				\left( \sin^2\theta_\mathrm{W} \right)^\mathrm{PDG} = \num{0.23126 +- 0.00005}
			\end{align*}
			
			
		\end{itemize}
\end{frame}




\section{Conclusion}

\begin{frame}
	\begin{center}
		\LARGE
		\textbf{Thank you for your attention!}
	\end{center}
\end{frame}

\begin{frame}{Bibliography}
	\scriptsize
	\begin{thebibliography}{9}
		\bibitem[PDG]{pdg}
			K.A. Olive \textit{et al.} (Particle Data Group),
			\emph{The Review of Particle Physics},
			Chin.\ Phys.\ C, \textbf{38}, 090001 (2014) and 2015 update.
	\end{thebibliography}
\end{frame}

\beginbackup

\begin{frame}{Error-Estimates from the Binomial Distribution}
	\begin{align*}
		\Delta n_i = \sqrt{N \cdot \left[ \frac{n_i}{N} - \left( \frac{n_i}{N} \right)^2 \right]}
	\end{align*}
\end{frame}

\backupend

\end{document}