% PAKETE UND DOKUMENTKONFIGURATION
\documentclass[11pt, a4paper]{article}

% Encoding für Umlaute
\usepackage[utf8]{inputenc}
\usepackage[T1]{fontenc}

% Silbentrennung
\usepackage[ngerman]{babel}

% erweiterte Matheumgebungen und Formelnummer mit Sectionnummer
\usepackage{amsmath}
\numberwithin{equation}{section}

% Braket Notation
\usepackage{braket}
\usepackage{isotope}
\usepackage[version=4]{mhchem}
\usepackage{tensor}
\usepackage{slashed}

% zusätzliche mathematische Schriftarten
\usepackage{amsfonts}

% verschiedene mathematische Symbole
\usepackage{amssymb}

% sidewaysfrac
\usepackage{xfrac}

% Einheiten setzen z.B. \SI{10}{\kilo\gram\meter\per\second\squared}
% Fehler: \SI{10 +- 0,2e-4}{\metre}
\usepackage{siunitx}
\sisetup{
  output-decimal-marker={,},
  separate-uncertainty
}

% Einheitendefinitionen
\DeclareSIUnit{\skt}{Skt.}
\DeclareSIUnit{\gauss}{G}
\DeclareSIUnit{\division}{div.}
\DeclareSIUnit{\Kanal}{Kanal}

% Operatordefinitionen
\DeclareMathOperator{\erf}{erf}

% Randbreiten
\usepackage[left=3.5cm,right=3.5cm,top=3cm,bottom=3cm,twoside]{geometry}

% Bilder einfügen
\usepackage{graphicx}
\usepackage[percent]{overpic}

% Textfarbe
\usepackage{color}

% Verweise innerhalb des Dokuments
\usepackage{hyperref}
\hypersetup{
	colorlinks = true,
	allcolors = {black}
}

% bessere Tabellenlayouts
\usepackage{booktabs}
\usepackage{multirow}
\usepackage{multicol}

% Seitenlayout (Kopfzeile)
\usepackage{fancyhdr}

% Float Barriers
\usepackage{placeins}

% Pakete für gedrehte Subfigures
\usepackage{caption}
\usepackage{subcaption}
\usepackage{rotating}
\usepackage{capt-of}

% Paket für textumflossene Abbildungen und Tabellen
\usepackage{wrapfig}

\usepackage{float}

% Caption-Setup
\captionsetup{font={small}}
\renewcommand{\thefigure}{\thesection.\arabic{figure}}
\renewcommand{\thesubfigure}{\alph{subfigure}}
\renewcommand{\thetable}{\thesection.\arabic{table}}
\renewcommand{\thesubtable}{\alph{subtable}}

% Manuelle Silbentrennung
\hyphenation{Sekundär-elek-tronen-verviel-facher}

% Tiefe des Inhaltsverzeichnisses (Level: 1 sections, 2 subsections,
% 3 subsubsections)
\setcounter{tocdepth}{3}

% FANCYHDR SETUP
\pagestyle{fancy}
\fancyhead[EL,OR]{\thepage}
\fancyhead[ER]{\leftmark}
\fancyhead[OL]{\rightmark}
\setlength{\headheight}{13.6pt}

\renewcommand{\sectionmark}[1]{
\markboth{\thesection{} #1}{\thesection{} #1}
}
\renewcommand{\subsectionmark}[1]{
\markright{\thesubsection{} #1}
}

\newcommand{\korr}[1]{{\color{red}(#1)}}

% DOKUMENTINFORMATIONEN
\title{A248 \\ Magneto-optische Falle}

\author{Christopher Deutsch\footnote{christopher.deutsch@uni-bonn.de} \and Christian Bespin\footnote{christian.bespin@uni-bonn.de}}

\date{\today}

\begin{document}

\begin{titlepage}

\maketitle

% DURCHFÜHRUNGSDATUM UND TUTOR
\begin{center}
\begin{tabular}{l r}
Durchführung: & 4./5. April 2016 \\
Gruppe: & P8 \\
Tutor: & Daniel Babik
\end{tabular}
\end{center}

% ZUSAMMENFASSUNG
\begin{abstract}
\noindent
\end{abstract}

\end{titlepage}

% INHALTSVERZEICHNIS
\tableofcontents
% Neue Seite nach TOC
\newpage

% INHALT VERSUCHSPROTOKOLL
\section{Einführung}

Das Ziel von Atomfallen ist die Speicherung von Atomen in einem begrenzten Raumgebiet, ohne den Einsatz von harten Wänden.
Um dies zu ermöglichen muss die thermische Bewegung der Atome minimiert werden, sodass gekühlt werden müssen.
Anschließend kann durch geeignete Mechanismen die Bewegung der Atome auf ein Raumgebiet beschränkt werden.
Ein solcher Aufbau kann durch eine sogenannte magneto-optische Falle realisiert werden, die in diesem Praktikumsversuch in Betrieb genommen und charakterisiert werden soll.

\section{Theorie}

\korr{Hier noch kurze Einführung}

\begin{figure}
	\centering
	\includegraphics{./figures/theory/streukraft.pdf}
	\caption{Streukraft}
\end{figure}

\subsection{Laserkühlung}
\korr{Wie ist der Zusammenhang zu dem Bild?}

Zum Kühlen der Atome wird Laserlicht genutzt und insbesondere der Impulsübertrag von Photonen auf Atome ausgenutzt.
Der Impuls $\hbar k$ eines Photons mit der Wellenzahl $k$ kann von einem Atom absorbiert werden.
Die aus der Absorption gewonnene Energie des Atoms kann dieses in ein energetisch höher gelegenes Niveau anregen, wobei die Richtung des bei der anschließenden spontanen Emission entstehenden Photons zufällig ist.
Für eine hohe Anzahl einfallender Photonen ist die Emission und der bei der Abregung entstehende Impulsübertrag isotrop, weswegen bei kontinuierlicher Bestrahlung durch einen Laser ein Nettoimpulsübertrag auftritt.

Ein sich bewegendes Atom nimmt aufgrund seiner Geschwindigkeit $v$ das Laserlicht wegen des Dopplereffekts mit einer um $kv$ verschobenen Frequenz gegenüber der im Laborsystem wahr.
Eine Kühlung ist daher nur möglich, wenn der Laser bezüglich des Energieabstands der beiden Niveaus, die zum Kühlen verwendet werden, rotverstimmt ist, das heißt eine geringere Frequenz hat.
Aufgrund des Dopplereffekts nimmt ein Atom, welches sich also entgegen der Richtung des Laserstrahls bewegt, das Licht des Laser mit einer Frequenz wahr, welche das Atom in den angeregten Zustand versetzt.
Durch die Absorption des Photons, deren Impulsübertrag immer in die gleiche Richtung zeigt und die isotrope Emission wird ein Atom, welches sich entgegen der Richtung des Laserstrahls bewegt, gekühlt.
Da die Rotverschiebung geschwindigkeitsabhängig ist, ist auch die resultierende kühlende Kraft auf das Atom proportional zur Geschwindigkeit, weswegen die Kühlung auch als optische Melasse bezeichnet wird.
Für ein Zweiniveausystem ist dies in Abbildung \ref{fig:opt_melasse} dargestellt.
\begin{figure}[h]
	\centering
	\includegraphics[width=.8\textwidth]{./figures/theory/melasse}
	\caption{Optische Melasse eines Zweiniveausystems. Links: Kraftwirkung auf ein ruhendes Atom. Rechts: der Bewegung eines Atoms entgegengesetzte Kraft.}
	\label{fig:opt_melasse}
\end{figure}

\subsection{Fangen von Atomen}
Im Folgenden wird nur ein vereinfachtes System betrachtet, da reale Atome eine Vielzahl von Energieniveaus besitzen, die die hier durchgeführte Betrachtung deutlich verkomplizieren.
Um die gekühlten Atome, die sich zwar mit geringerer Geschwindigkeit aber weiterhin frei bewegen, an einem bestimmten Punkt zu konzentrieren wird der Zeeman-Effekt genutzt.
Dieser bewirkt eine Aufspaltung der atomaren Niveaus in drei Zustände mit $m_J = 0, \pm1$, deren Energieverschiebung vom Zustand mit $m_J=0$, was dem ursprünglichen Niveau entspricht, von der Stärke des Magnetfeldes abhängt:
\begin{align}
	\Delta E =\mu_\mathrm{B}B
\end{align}
mit dem Bohrschen Magneton $\mu_\mathrm{B}$.
Übergänge in diese Zustände können durch links- bzw. rechtszirkular polarisiertes Licht gezielt ausgewählt werden.

Für das Magnetfeld wird ein Spulenaufbau in Anti-Helmholtz-Konfiguration verwendet, der ein linear ansteigendes Magnetfeld in alle Richtungen ermöglicht.
Im Zentrum des Aufbaus, in dem die Atome gefangen werden, verschwindet das Magnetfeld, nicht aber der Gradient.
Damit die Atome die Falle nicht verlassen werden, wie in Abbildung \ref{fig:magnetfeld} dargestellt, aus entgegengesetzten Richtungen jeweils ein linkszirkular und ein rechtszirkular polarisierter Laser eingestrahlt.
\begin{figure}[h]
	\centering
	\includegraphics[width=.8\textwidth]{./figures/theory/mot.pdf}
	\caption{Nutzung des Zeeman-Effekts, um eine ortsabhängige Kraft auf die zu fangenden Teilchen auszuüben.\korr{Bild ändern, Beschriftungen vergrößern}}
	\label{fig:magnetfeld}
\end{figure}
Für ein nach links ausgelenktes Atom, d.h. im Bereich der Magnetfeldstärke $B<0$ liegt der Zustand mit $m_J=+1$ unter dem Niveau mit $m_J=0$, so dass ein diesem Energieniveau gegenüber rotverstimmter Laser mit $\sigma^+$ Polarisation mit dem $m_J=+1$ Zustand wechselwirkt und eine Kraft in Richtung des Zentrum der Falle auf das Atom ausübt.
Der analoge Fall gilt für ein nach rechts ausgelenktes Atom im Bereich von $B>0$, wo der von rechts eingestrahlte Laser mit $\sigma^-$ Polarisation mit dem $m_J=-1$ Zustand interagiert, der ebenfalls unterhalb des ursprünglichen Energieniveaus liegt.
Somit erhält man durch die gezielte Einstrahlung von polarisiertem Licht, welches entsprechend dem Magnetfeld polarisiert sein muss, eine ortsabhängige Kraft auf die Atome, die sie in das Zentrum der Falle drückt und somit ein Auseinanderlaufen der gekühlten Atome verhindert.

\subsection{Übergänge in Rubidium}
\label{sec:rb_uebergaenge}

In diesem Versuch soll das Rubidiumisotop \isotope[85]{Rb} eingefangen werden.
Dazu ist es nötig, detaillierte Informationen über das Niveauschema und die Übergänge dieser Atome zu haben, um den optimalen Übergang zum Kühlen und Fangen der Atome auszuwählen.
Es eignet sich der Übergang $F=3 \rightarrow F^\prime=4$, da dieser beinahe abgeschlossen ist \korr{das ist abgeschlossen nicht nur beinahe}, d.h. dass das Atom durch Abregung vom Zustand $F^\prime=4$ beinahe ausschließlich in den Zustand $F=3$ übergeht \korr{das geht immer in den zustand über}.
Dieser Übergang entspricht einer Wellenlänge von \SI{780.25}{nm}, so dass ein entsprechender Laser zur resonanten Anregung gewählt wird \korr{Nicht Resonant -> Detuning}.
Da jedoch in einigen seltenen Fällen auch eine Anregung in die Niveaus $F^\prime=2$, $F^\prime=3$ möglich ist, die in den $F=2$ Zustand übergehen können und die Atome so das abgeschlossene System zur Kühlung verlassen, wird zusätzlich ein sogenannter Pumplaser eingesetzt, der die Atome aus diesem Zustand auf das $F^\prime=3$ Niveau anhebt, von wo aus sie wieder am Kühlprozess teilnehmen können\korr{Die nehmen noch nicht am Kühlprozess teil. Müssen erst wieder in das untere Niveau des Kühlübergangs zerfallen und das passiert nicht immer.}.
\korr{Niveauschema hierhin} 

\subsection{Lasersystem}

Wie in Abschnitt \ref{sec:rb_uebergaenge} erwähnt, werden zum Betrieb der magneto-optischen Falle zwei Laser benötigt, die auf die entsprechenden Frequenzen für den Kühl- und Pumplaser eingestellt werden müssen.
Hierzu werden zwei Diodenlaser in Littrow-Konfiguration genutzt, d.h. das emittierte Laserlicht wird kollimiert und trifft auf ein Reflexionsgitter, welches so in den Strahlengang gebracht wird \korr{Ich glaube das ist egal wie es steht. Die erste Ordnung geht immer zurück in den Laser}, dass die Beugung der~-1.~Ordnung zurückreflektiert wird und somit einen externen Resonator bildet, während das Licht, welches in~0.~Ordnung gebeugt wird, ausgekoppelt wird.
Durch ein elektrisch anteuerbares Piezo-Element kann das Gitter gedreht und somit die Resonatorlänge und damit die Frequenz des ausgekoppelten Lichts geändert werden.
\korr{evtl Bild von Littrow}

Da die nötigen Frequenzen möglichst genau auf die Übergänge im Rubidium abgestimmt sein müssen, wird ein Teil des Laserlichts zu Spektroskopiezwecken mit Rubidium verwendet.
Als Spektroskopieverfahren kommt die Polarisationsspektroskopie zum Einsatz, die eine Erweiterung der dopplerfreien Sättigungsspektroskopie darstellt.
Dabei wird der Lichtstrahl in einen Pumpstrahl mit hoher Intensität und einen Teststrahl mit geringerer Intensität aufgespalten, die in ungefähr entgegengesetzter Richtung eine Kammer mit Rubidiumatomen durchlaufen.
Während der Pumplaser den gewählten Übergang sättigt, d.h. möglichst viele Atome in den angeregten Zustand bringt, wird die Intensität des Teststrahls beobachtet.
Diese ist hoch, wenn der Übergang vollständig durch den Pumpstrahl gesättigt ist und gering, wenn Atome durch den Teststrahl angeregt werden \korr{Hier vllt ne kurze Erklärung}.
In Abhängigkeit von der Frequenz aufgetragen ergibt sich ein Intensitätsverlauf wie in Abbildung \korr{ref}.
\korr{Bild von Sättigungssignal}
Dieses lässt sich dadurch erklären, dass die Atome Raumtemperatur haben und sich somit sehr schnell bewegen.
Aufgrund der Dopplerverschiebung interagieren die Laser immer mit Atomen einer bestimmten Geschwindigkeitsklasse, abhängig von der Bewegungsrichtung der Atome relativ zum Laserstrahl.
Diese Bewegung ist für die Dopplerverschiebung verantwortlich und erklärt die Abflachung des Signals.
Das Maximum, auch Lamb-Dip genannt, ergibt sich durch Atome, die sich in einer Ebene senkrecht zu den Laserstrahlen bewegen.
Die Dopplerverschiebung verschwindet hier, so dass der Pumpstrahl und Teststrahl mit der gleichen Frequenz im Ruhesystems des Atoms strahlen und der Übergang somit gesättigt wird \korr{durch den Pumpstrahl}, weswegen der Teststrahl keinen Intensitätsverlust erfährt.
Die natürliche Linienbreite kann auf diese Weise beobachtet werden.

In der Polarisationsspektroskopie wird der Pumpstrahl zirkularpolarisiert, so dass die $m_J=\pm1$ \korr{Beide oder nur einer? -> nur einer} Niveaus gesättigt werden.
Diese anisotrope Besetzung ist dafür verantwortlich, dass das Rubidiumgas doppelbrechend wird und die lineare Polarisation des Teststrahls beim Durchgang um einen Winkel $\alpha$ gedreht wird.
\korr{Änderung des Brechungsindex für was? Unterschiedliche Brechungsindizes für die verschiedenen Zirkularpolarisationen -> optische Aktivität.}
Dieser Winkel ist abhängig von der Änderung des Brechungsindex, die über die Kramers-Kronig-Relation mit dem Absorptionsprofil zusammenhängt \korr{scheiß Relation irgendwie erläutern} \korr{Ich seh selber keine Relevanz der Relation für unser Protokoll -> raus damit}.
Durch einen Polarisator und einen Analysator \korr{Das ist der PBS} vor und hinter der Rubidiumkammer kann detektiert werden, ob die Polarisation des Teststrahls gedreht wurde.
Dazu wird der Strahl durch einen polarisierenden Strahlteiler geteilt, wobei beide Strahlen das bekannte Spektrum der Sättigungsspektroskopie aufweisen.
Durch Subtraktion der beiden Spektren kann ein Signal generiert werden \korr{Was ist das gute an dem Signal? Vorallem im Vgl.\ zur Sättigungsspek.}, mit dem eine Rückkopplungsschleife betrieben wird \korr{Lockbox erwähnen}, die das Piezoelement am Reflexionsgitter ansteuert, um die Frequenz des Lasers anzupassen.
Dadurch ist es möglich, die Frequenz des Lasers mit hoher Genauigkeit einzustellen und Temperaturdrifts und anderen äußeren Einflüssen, die eine Änderung der Frequenz hervorrufen, entgegenzuwirken.

\section{Aufbau der MOT}

\subsection{Einstellung der Laserfrequenzen}

Zunächst wurden sowohl der Pump- als auch der Kühllaser auf die richtigen Frequenzen eingestellt.
Hierzu ist es wichtig, in den auf dem Oszilloskop beobachtbaren Spektren die richtigen Übergänge im Rubidium zu identifizieren, auf die die Laser eingestellt werden sollen.
Als Hilfe dienten die dargestellten Spektren in \cite{anleitung} für das \isotope[85]{Ru} Isotop.
Der Kühllaser wurde auf den Übergang $F=3 \rightarrow F^\prime=4$ eingestellt, indem durch Änderung der Scanamplitude und des Scanoffsets das in der Polarisationspsketroskopie beobachtete Fehlersignal einen Nulldurchgang bei einer leichten Rotverschiebung gegenüber des gewünschten Rubidiumübergangs zeigt.
Durch fixieren dieses Signals mit dem Lock-Schalter an der Lockbox wird die Rückkopplungsschleife betrieben, die das Gitter am Diodenlaser ansteuert und damit die Frequenz reguliert.

Der Pumplaser wurde auf die gleiche Weise auf den $F=2 \rightarrow F^\prime=3$ Übergang eingestellt, jedoch kann dieser Laser nicht über eine Rückkopplungsschleife auf eine Frequenz fixiert werden \korr{Wir haben beide Übergänge ->2 und ->3 im im Scan. Die 2 kann auch nach 3 zerfallen und ist somit im Kühlkreislauf}.
Da dieses Signal jedoch nicht mit der gleichen Genauigkeit wie beim Kühllaser eingestellt werden muss, ist ein Betrieb der Falle trotzdem möglich.

\subsection{Justage der Laser an der Vakuumkammer}

Mit einem Leistungsmessgerät wurde zunächst überprüft, ob genügend Licht aus der optischen Faser austritt, wobei eine Leistung von \SI{14,5 +- 0.1}{mW} festgestellt werden konnte.
Durch zwei Strahlteiler wird das Laserlicht in drei Strahlen geteilt, die jeweils senkrecht zueinander stehen und die Vakuumkammer durchlaufen, bevor sie von einem Spiegel in sich selbst reflektiert werden, so dass insgesamt sechs Strahlen die Kammer auf drei Achsen passieren.
Zwei der Strahlen verlaufen in der horizontalen Ebene, einer entlang der vertikalen Achse.
Durch $\lambda/2$-Platten kann die Leistung der einzelnen Teilstrahlen individuell eingestellt werden, wobei alle Strahlen ungefähr die gleiche Leistung haben sollten.
Für den vertikalen Strahl wurde eine Leistung vo \SI{3,0 +- 0.1}{mW} eingestellt, für die beiden horizontalen \SI{3,0 +- 0.1}{mW} bzw. \SI{3,2 +- 0.1}{mW}.
Da diese Leistungen sich nicht zu den ursprünglich zur Verfügung stehenden \SI{14,5}{mW} addieren, kann festgestellt werden, dass in dem Aufbau eine Lichtleistung von ca. \SI{5,3}{mW} verloren geht.

Im Anschluss wurden die Aufbauten, d.h.\ die Spiegel und Einstellung der $\lambda/4$ Platten optimiert, so dass sich auf den drei Achsen sowohl einfallender als auch reflektierter Laserstrahl überlagern und sich die Strahlen im Zentrum der Vakuumkammer kreuzen.
Nach Einschalten des Magnetfeldes konnte mit der CCD-Kamera bereits eine Fluoreszenz und eine kleine Wolke gefangener Rubidiumatome erkennen.

\subsection{Optimierung der gefundenen Einstellungen}

Während eine Änderung der Rotverschiebung des Kühllasers nur eine geringe Auswirkung auf die Größe und Intensität der beobachteten Atome zeigte, konnte vor allem durch eine Feinjustage der Frequenz des Pumplasers die Falle optimiert werden.
Ein zusätzliches konstantes Magnetfeld verschlechterte das beobachtete Bild, so dass diese Möglichkeit zur Optimierung nicht genutzt wurde und im Folgenden nur die Frequenz des Pumplasers geändert wurde.
Nach Anbringen eines Leistungsmessgeräts zur Bestimmung der Fluoreszenzleistung der Falle konnte auch quantitativ eine Optimierung vorgenommen werden.
So konnte eine Leistung von \SI{114 +- 5}{nW} erreicht werden, wobei dieser Wert aufgrund von Temperaturschwankungen im Raum und daher Änderungen der Laserfrequenz \korr{Welcher Laser? Repump -> nicht gelockt} mit der Zeit eine leichte \korr{die waren schon relativ stark} Veränderung erfuhr.
Nachdem sich näherungsweise ein thermisches Gleichgewicht eingestellt hatte, war diese Veränderung jedoch sehr gering, so dass die folgenden Messungen durchgeführt werden konnten, wobei gegebenenfalls zwischen zwei Messungen eine Nachjustierung erfolgte.

\section{Charakterisierung der MOT}
\label{sec:charakterisierung_mot}
Nachdem die MOT aufgebaut und auf möglichst hohe Fluoreszenz optimiert wurde, kann im Folgenden die Charakterisierung einiger Eigenschaften der Falle durchgeführt werden.

\subsection{Atomzahl in der MOT}
Im Folgenden soll die Anzahl der gefangenen Atome in der MOT bestimmt werden.
Dazu muss zunächst der Fluoreszenzleistung der gefangenen Atome und der Strahlradius des Lasers bestimmt werden.

\subsubsection{Bestimmung der Fluoreszenz der MOT}
\label{sec:fluoreszenz}
Zur Bestimmung der Fluoreszenz der MOT wird die leuchtende Atomwolke mithilfe einer Linse auf ein Leistungsmessgerät abgebildet.
Dabei gilt es zu beachten, dass die so gemessene Leistung auch die Fluoreszenz von ungefangenen Atomen beinhaltet.
Daher wird nach jeder Messung das Magnetfeld der Falle deaktiviert um eine Hintergrundmessung durchzuführen und der so gemessene Hintergrund kann vom Fluoreszenzsignal abgezogen werden.
Somit wird die Fluoreszenzleistung, die die Atomwolke in dem von der Linse aufgespannten Raumwinkel emittiert, zu
\begin{align*}
	P_\mathrm{Linse} = \SI{114 +- 5}{nW}
\end{align*}
bestimmt.
Um die Fluoreszenzleistung zu erhalten, die im gesamten Raumwinkel von~$4\pi$ emittiert wird, kann
\begin{align*}
	P_\mathrm{tot} = P_\mathrm{Linse} \cdot \frac{A_\mathrm{tot}}{A_\mathrm{Linse}}
\end{align*}
genutzt werden.
Dabei beschreibt $A_\mathrm{Linse}$ die Fläche der Linse und $A_\mathrm{tot}$ die Oberfläche der Kugel mit einem Radius $R$, der dem Abstand der MOT zur Linse entspricht.
Somit erhält man
\begin{align*}
	P_\mathrm{tot} = P_\mathrm{Linse} \cdot \frac{4 \pi R^2}{\frac{1}{4} \pi d_\mathrm{Linse}^2} = P_\mathrm{Linse} \cdot \frac{16 R^2}{d_\mathrm{Linse}^2}
\end{align*}
und mit den gemessenen Werten für den Linsendurchmesser
\begin{align*}
	d_\mathrm{Linse} = \SI{2.5 +- 0.1}{cm} \, \text{,}
\end{align*}
sowie dem Abstand~$R$ zwischen MOT und Linse
\begin{align*}
	R = \SI{10.3 +- 0.2}{cm}
\end{align*}
kann die gesamte Leistung berechnet werden.
Diese ergibt sich zu
\begin{align*}
	P_\mathrm{tot} = \SI{30.9 +- 3.1}{\micro\watt} \, \text{,}
\end{align*}
wobei der Fehler durch \textsc{Gauß}sche Fehlerfortpflanzung gemäß
\begin{align*}
	\Delta P_\mathrm{tot} = \frac{16}{d_\mathrm{Linse}^3} \sqrt{4 R^2 P_\mathrm{Linse}^2 d_\mathrm{Linse}^2 \Delta R^2 + R^4 \left( d_\mathrm{Linse}^2 \Delta P_\mathrm{Linse}^2 + 4 P_\mathrm{Linse}^2 \Delta d_\mathrm{Linse}^2 \right)}
\end{align*}
berechnet wurde.


\subsubsection{Bestimmung des Strahlradius}
\label{sec:strahlradius}
Anschließend muss eine Bestimmung des Radius des Laserstrahls durchgeführt werden.
Dazu wird das Leistungsmessgerät in den Strahlengang gestellt und der Laser wird seitlich durch eine Messerklinge blockiert.
Durch eine Messung der Leistung des teilweise blockierten Laserstrahls in Abhängigkeit der Position der Klinge kann der Strahlradius bestimmt werden.
Dazu wird angenommen, dass der Laserstrahl durch einen \textsc{Gauß}-Strahl beschrieben werden kann, weshalb die Intensität des Strahls in der transversalen Ebene durch
\begin{align*}
	I(x, y) = \frac{2 P_0}{\pi w^2} \exp\left(-\frac{2 (x^2 + y^2)}{w^2} \right)
\end{align*}
mit der Gesamtleistung des Strahls~$P_0$ und dem Strahlradius~$w$, das ist die Entfernung zum Strahlmittelpunkt bei dem die Intensität auf $\sfrac{1}{e^2}$ abgefallen ist, gegeben ist.
Durch Integration kann nun die Leistung des teilweise blockierten Strahls bestimmt werden
\begin{align}
	P(x) &= \int_{x}^{\infty} \mathrm{d}x^\prime \int_{-\infty}^{\infty} \mathrm{d}y^\prime \, I(x^\prime, y^\prime) \\
	     &= \frac{P_0}{2} \left[ 1 - \erf\left( \frac{\sqrt{2} x}{w}\right) \right] \, \text{,}
	\label{eq:erf_fit}
\end{align}
wobei $x$ den von der Klinge blockierten Teil des Strahls beschreibt.

\begin{figure}
	\centering
	\includegraphics{./figures/beam_radius_fit.pdf}
	\caption{Anpassung einer Funktion gemäß der Hypothese in Gleichung~\eqref{eq:erf_fit} an die transmittierten Leistungen~$P$ nach der Blockierung des Strahlengangs mit einer Klinge an der Position~$x$. Der Fehler der Leistung~$\Delta P$ liegt in der Größenordnung der Markierung.}
	\label{fig:beam_radius}
	
	\vspace{1cm}
	
	\begin{tabular}{S[table-format=1.1]S[table-format=1.1]S[table-format=1.2]S[table-format=1.2]}
\toprule
{$x$ / \si{mm}} & {$\Delta x$ / \si{mm}} & {$P$ / \si{mW}} & {$\Delta P$ / \si{mW}} \\
\midrule
            0.0 &                    0.2 &            2.85 &                   0.02 \\
            1.0 &                    0.2 &            2.84 &                   0.02 \\
            2.0 &                    0.2 &            2.84 &                   0.02 \\
            3.0 &                    0.2 &            2.81 &                   0.02 \\
            4.0 &                    0.2 &            2.74 &                   0.02 \\
            5.0 &                    0.2 &            2.62 &                   0.02 \\
            6.0 &                    0.2 &            2.25 &                   0.02 \\
            7.0 &                    0.2 &            1.62 &                   0.02 \\
            8.0 &                    0.2 &            0.83 &                   0.02 \\
            9.0 &                    0.2 &            0.26 &                   0.02 \\
           10.0 &                    0.2 &            0.01 &                   0.02 \\
           11.0 &                    0.2 &            0.00 &                   0.02 \\
\bottomrule
\end{tabular}

	\captionof{table}{Transmittierte Leistung des Strahls~$P$ in Abhängigkeit der Klingenposition~$x$ zur Bestimmung des Strahlradius~$w$.}
	\label{tab:beam_radius}	
\end{figure}

Da in der Praxis keine relative Messung von Klingenposition~$x$ zum Strahlmittelpunkt durchgeführt werden kann, wird stattdessen eine Anpassung der um $\mu$ verschobenen Funktion $P(x-\mu)$ an die gemessenen Leistungen durchgeführt, um aus der Messung die Strahlleistung~$P_0$, den Strahlradius~$w$ und den Strahlmittelpunkt~$\mu$ zu extrahieren.
Die Anpassung einer solchen Kurve an die gemessenen Werte aus Tabelle~\ref{tab:beam_radius} 
führt zu den Anpassungsparametern:
\begin{align*}
	P_0 &= \SI{2.818 +- 0.017}{mW} \\
	w &= \SI{2.826 +- 0.089}{mm} \\
	\mu &= \SI{7.217 +- 0.033}{mm} \, \text{.}
\end{align*}
Die Messpunkte sowie die Anpassung wurden in Abbildung~\ref{fig:beam_radius} dargestellt.
Durch die Anpassung wurde demnach der Strahlradius~$w$ zu \SI{2.826 +- 0.089}{mm} bestimmt und kann folglich für die Auswertung verwendet werden.

\subsubsection{Streurate}
Weiterhin muss die Streurate~$R$ des Kühllasers mit einer Verstimmung~$\delta$ vom Kühlübergang mit natürlicher Linienbreite~$\Gamma$ bestimmt werden.
Diese ist gegeben durch
\begin{align*}
	R = \frac{\Gamma}{2} \frac{I / I_\mathrm{S}}{1 + I / I_\mathrm{S} + 4 \delta^2 / \Gamma^2}
\end{align*}
mit der Intensität des Lasers~$I$ und der Sättigungsintensität~$I_\mathrm{S}$ des Übergangs \cite{foot}.
Im Folgenden wird der Strahlradius~$w$, welcher in Abschnitt~\ref{sec:strahlradius} bestimmt wurde, genutzt um die Intensität des Kühlstrahls abzuschätzen.
Dabei wird die Gesamtleistung des Kühllasers berechnet durch
\begin{align*}
	P_\mathrm{K"uhl.} &= 2 \cdot \left[ \SI{3.0 +- 0.1}{mW} + \SI{3.0 +- 0.1}{mW} + \SI{3.2 +- 0.1}{mW} \right]\\
	  &= \SI{18.4 +- 0.4}{mW} \, \text{,}
\end{align*} 
wobei der Faktor 2 aufgrund der Rückreflexion des Kühlstrahls entsteht.
Nimmt man an, dass diese Leistung gleichmäßig auf einem Kreis mit Radius~$w$ verteilt ist, so erhält man für die Intensität mit dem Ergebnis für den Strahlradius aus Abschnitt~\ref{sec:strahlradius}
\begin{align*}
	I = \frac{P_\mathrm{K"uhl.}}{\pi w^2} = \SI{73.3 +- 4.9}{\milli\watt\per\centi\meter\squared} \, \text{.}
\end{align*}
Unter der Verwendung der natürlichen Linienbreite~$\Gamma = \SI{38.117+-0.011}{MHz}$, der Sättigungsintensität~$I_\mathrm{S} = \SI{1.66932 +- 0.00052}{\milli\watt\per\centi\meter\squared}$ aus \cite{steck}, sowie der Verstimmung des Kühllasers~$\delta = \SI{-11.6+- 0.4}{MHz}$ (wird in Abschnitt~\ref{sec:detuning_cooling} bestimmt) erhält man die Streurate
\begin{align*}
	R = \SI{18.5 +- 0.1}{MHz} \, \text{.}
\end{align*}
Der Fehler wurde dabei durch \textsc{Gauß}sche Fehlerfortpflanzung berechnet.

\subsubsection{Bestimmung der Anzahl der gefangenen Atome}
\label{sec:atomzahl}
Schließlich sind alle nötigen Größen bekannt und eine Berechnung der Anzahl der gefangenen Atome kann durchgeführt werden.
Diese kann aus der gesamten Fluoreszenzleistung der gefangenen Atome~$P_\mathrm{tot}$ (Bestimmung in Abschnitt \ref{sec:fluoreszenz}), der Streurate~$R$ der Photonen des Kühllasers (Bestimmung im vorigen Abschnitt) und der bekannten Energie des Übergangs $\hbar \omega = \SI{1.589}{\electronvolt}$ \cite{steck} berechnet werden.
Somit folgt die Anzahl der gefangenen Atome
\begin{align*}
	N = \frac{P_\mathrm{tot}}{\hbar \omega R}
\end{align*}
und mit den zuvor berechneten Größen
\begin{align*}
	N = \num{6.6 +- 0.7e6} \, \text{.}
\end{align*}
Der Vergleich mit den in \cite{wieman} erreichten \num{4e7} Atomen zeigt, dass die Größenordnung der Anzahl der gefangenen Atomen übereinstimmt.
Mit zusätzlicher Feinjustage und einer funktionstüchtigen Frequenzstabilisierung für den Pumplaser sollte demnach die Größenordnung von $10^7$ Atomen mit dem vorliegenden Aufbau zu leicht erreichen sein.


\subsection{Größe der MOT}
Zur Bestimmung der Größe wird mit der CCD-Kamera ein Bild der in der Falle gefangenen Atome erstellt.
Um die Größe anzugeben, wurde im Anschluss bei gleicher Brennweite eine Millimeterskala im gleichen Abstand wie die Falle aufgenommen.
Dadurch konnte der Längenmaßstab der Aufnahmen zu~$\SI{1}{px}=\SI{0,0377+-0,0005}{mm}$ \korr{evtl.\ Bild von der Skala} bestimmt werden.
Entsprechend beträgt die Fläche eines Pixels dann~\SI{1.421+-0.038e-3}{mm^2}.
Mit einem Bildbearbeitungsprogramm wurden in dem aufgenommenen Bild mehrere Abstände gegenüberliegender Punkte auf dem Rand des markierten Bereichs gemessen, da die beobachtete Atomwolke keiner einfachen geometrischen Figur ähnelt, deren Durchmesser einfach berechnet werden könnte.
Die gemessenen Strecken wurden im Anschluss gemittelt und so ein Durchmesser von ungefähr \SI{1,47+-0,12}{mm} ermittelt werden, wobei der Fehler aus der Varianz folgt.
Durch eine Zählung der Pixel konnte weiterhin die Fläche zu \SI{1,16+-0,2}{mm^2} bestimmt werden.
Der Fehler wurde in diesem Fall abgeschätzt aufgrund von einer möglicherweise nicht idealen Auswahl des zu vermessenden Bereichs und entspricht einer Abweichung von~\num{162} Pixeln.
In Abbildung \ref{fig:mot_groesse} ist der untersuchte Bereich zur Bestimmung von Durchmesser und Fläche zu sehen.
\begin{figure}[h]
	\centering
	\includegraphics[width=.7\textwidth]{./figures/size_inverted.png}
	\caption{Zur Auswertung genutzter Bereich im mit der CCD-Kamera aufgenommenen Bild. Für eine bessere Druckqualität wurden die Farben invertiert.}
	\label{fig:mot_groesse}
\end{figure}

\subsection{Einfluss der $\lambda / 4$-Platten}
Im Folgenden soll der Einfluss der $\lambda / 4$-Platten auf die Population der Falle untersucht werden.
Dazu wird erneut die MOT durch eine Linse auf das Leistungsmessgerät abgebildet, um ein Fluoreszenzsignal zu messen.
Die im Aufbau auftretenden $\lambda / 4$-Verzögerungsplatten lassen sich in zwei Gruppen einteilen.
Diese sind die Platten für den in die Falle eingehenden Strahl und die für den reflektierten Strahl direkt am retroreflektierenden Spiegel.
Da der Einfluss der einzelnen Platten für die jeweilige Fallenachse identisch ist, wird im Folgenden nur eine Achse detailliert untersucht.

\subsubsection{Verzögerungsplatte des eingehenden Strahls}
\label{sec:lambda_4_inc}
Zur Untersuchung des Einflusses der Verzögerungsplatte des eingehenden Strahls wird diese Schrittweise gedreht und sowohl der Stellungswinkel~$\varphi$ sowie die Fluoreszenzleistung~$P$ (mit subtrahiertem Untergrund) gemessen.
Dabei wird der Stellungswinkel schrittweise von \SI{0}{\degree} bis \SI{180}{\degree} in \SI{10}{\degree}-Schritten verstellt, bis ein gesamter Winkelbereich von \SI{180}{\degree} vermessen wurde.
Die somit gemessene Fluoreszenzleistung in Abhängigkeit des Drehwinkels der Platte für den eingehenden Strahl wurde in Abbildung \ref{fig:lambda_4_inc} aufgetragen.
Die gemessenen Werte wurden in Tabelle \ref{tab:lambda_4} zusammengefasst.
\begin{figure}[h]
	\centering
	\includegraphics{./figures/lambda_4_in.pdf}
	\caption{Einfluss des Stellwinkels $\varphi$ der $\lambda / 4$-Platte des eingehenden Strahls auf die Fluoreszenz $P$ der MOT.}
	\label{fig:lambda_4_inc}
\end{figure}

Man sieht, dass die Stellung der $\lambda / 4$-Platte des eingehenden Strahls einen starken Einfluss auf die Fluoreszenzleistung zeigt.
Dies liegt daran, dass diese Platte den eintreffenden linear-polarisierten Strahl in die für die Fallenoperation notwendige zirkulare Polarisation umwandelt.
Dazu muss die Achse der Verzögerungsplatte um $\pm\SI{45}{\degree}$ gegen die Polarisation des eintreffenden Strahls verkippt sein.
Darüber hinaus muss die richtige Zirkularpolarisation für diese Platte eingestellt werden, um den gemäß Abschnitt \korr{Theorie} erforderlichen Übergang zu treffen, der eine Operation der Falle ermöglicht.

In Abbildung \ref{fig:lambda_4_inc} sieht man, dass die Fluoreszenzleistung ein breites Minimum um einen Winkel der Verzögerungsplatte von \SI{80}{\degree} aufweist.
An dieser Stelle weist der Strahl gerade die falsche Zirkularpolarisation für das Fangen der Rubidiumatome auf.
Demnach sollte ein Drehen der Verzögerungsplatte um \SI{90}{\degree} nun die entgegengesetzte Zirkularpolarisation erzeugen und somit den Fallenbetrieb ermöglichen.
Dies entspricht dem breiten Maximum bei etwa \SI{170}{\degree}, bei dem der korrekte Übergang zum Fangen der Atome angeregt wird und somit den Fallenbetrieb ermöglicht.
Der auf der Verzögerungsplatte angegebene Optimalwert von \SI{172}{\degree} kann somit verifiziert werden.

Nach dem Abschluss dieser Messung werden die drei $\lambda / 4$-Platten für die eingehenden Strahlen per Hand auf das Maximum der Fluoreszenzleistung eingestellt.
Die Vergleich der Einstellung der Platten auf den verschiedenen Achsen~$\varphi$ mit den angegebenen Optimalwerten~$\varphi^\mathrm{opt}$ zeigt eine gute Übereinstimmung:
\begin{alignat*}{2}
\varphi_x &= \SI{180}{\degree} \qquad &&\varphi_x^\mathrm{opt} = \SI{172}{\degree}\\
\varphi_y &= \SI{214}{\degree} \qquad &&\varphi_y^\mathrm{opt} = \SI{224}{\degree}\\
\varphi_z &= \SI{80}{\degree} \qquad &&\varphi_z^\mathrm{opt} = \SI{80}{\degree} \, \text{.}
\end{alignat*}


\subsubsection{Verzögerungsplatte des reflektierten Strahls}
Letztlich soll der Einfluss der Verzögerungsplatte des reflektierten Strahls auf die Fluoreszenzleistung der MOT untersucht werden.
Dabei wird analog zu Abschnitt \ref{sec:lambda_4_inc} vorgegangen und die Fluoreszenzleistung~$P$ gegen den Winkel der Platte~$\varphi$ aufgetragen.
Die gemessenen Werte finden sich in Tabelle \ref{tab:lambda_4} und die Darstellung der Fluoreszenzleistung in Abbildung \ref{fig:lambda_4_out}.
\begin{figure}[h]
	\centering
	\includegraphics{./figures/lambda_4_out.pdf}
	\caption{Einfluss des Stellwinkels $\varphi$ der $\lambda / 4$-Platte des reflektierten Strahls auf die Fluoreszenz $P$ der MOT. Die Anpassung einer konstanten Funktion ergibt für die Fluoreszenzleistung $P_0 = \SI{124.4 +- 1.3}{nW}$.}
	\label{fig:lambda_4_out}
\end{figure}

Man sieht, dass die Fluoreszenzleistung kaum vom Winkel der Verzögerungsplatte abhängt.
Der Grund liegt darin, dass beim Eintreffen des zirkular-polarisierten Strahls aus dem Fallenzentrum auf die $\lambda / 4$-Platte dieser linear polarisiert wird.
Anschließend wird der linear-polarisierte Strahl am Spiegel reflektiert und trifft unter dem gleichen Winkel erneut auf die $\lambda / 4$-Platte und ist somit unabhängig von deren Stellwinkel.
Dies hat den Einfluss, dass die Helizität der Photonen umgekehrt wird und somit aus dem eingehenden Strahl mit $\sigma^\pm$-Polarisation ein rückläufiger Strahl mit $\sigma^\mp$-Polarisation entsteht.

Die Anpassung einer konstanten Funktion an die gemessene Fluoreszenzleistung ergibt
\begin{align*}
P_0 = \SI{124.4 +- 1.3}{nW}
\end{align*}
für die Fluoreszenzleistung der MOT.

\begin{table}
	\centering
	\begin{tabular}{S[table-format=3.0]S[table-format=3.0]S[table-format=3.0]}
\toprule
{$\varphi$ / \si{\degree}} & {$P_\mathrm{ein.}$ / \si{nW}} & {$P_\mathrm{ref.}$ / \si{nW}} \\
\midrule
                         0 &                           122 &                           118 \\
                        10 &                           115 &                           116 \\
                        20 &                           115 &                           114 \\
                        30 &                            97 &                           115 \\
                        40 &                            27 &                           124 \\
                        50 &                             5 &                           136 \\
                        60 &                            -1 &                           130 \\
                        70 &                             0 &                           128 \\
                        80 &                             1 &                           128 \\
                        90 &                             0 &                           123 \\
                       100 &                            -2 &                           123 \\
                       110 &                             0 &                           126 \\
                       120 &                            10 &                           127 \\
                       130 &                            49 &                           127 \\
                       140 &                            93 &                           125 \\
                       150 &                           110 &                           127 \\
                       160 &                           110 &                           128 \\
                       170 &                           114 &                           125 \\
                       180 &                           117 &                           123 \\
\bottomrule
\end{tabular}

	\caption{Gemessene Fluoreszenz der gefangenen Atome für verschiedene Stellungen der $\lambda / 4$-Platte des eingehenden Strahls~$P_\mathrm{ein.}$ und des reflektierten Strahls~$P_\mathrm{ref.}$. Auf die Fluoreszenzleistung wird ein Fehler von $\Delta P = \SI{5}{nW}$ angenommen.}
	\label{tab:lambda_4}
\end{table}


\subsection{Einfluss des Magnetfeldes}
Anschließend findet eine Messung der Fluoreszenzleistung der gefangenen Atome in Abhängigkeit des Stroms durch das Anti-Helmholtz-Spulenpaar statt.
Dieser Strom ist proportional zum Magnetfeldgradienten am Ort der Falle und hat somit einen Einfluss auf die Tiefe des Fallenpotentials.
Die gemessenen Werte der Fluoreszenzleistung~$P$ wurden in Abbildung~\ref{fig:magnetic_field} gegen den Spulenstrom~$I$ aufgetragen.
Darüber hinaus finden sich die gemessenen Werte in Tabelle~\ref{tab:magnetic_field}.
\begin{figure}[hp]
	\centering
	\includegraphics{./figures/magnetic_field.pdf}
	\caption{Einfluss des Spulenstroms des Anti-Helmholtz-Spulenpaares und somit des Magnetfeldgradienten im Zentrum der Falle auf die Fluoreszenzleistung der in der MOT gefangenen Atome.}
	\label{fig:magnetic_field}
	\vspace{1cm}
	\begin{minipage}[t]{0.3\textwidth}
		\centering
		\vspace*{-\dimexpr\baselineskip+\heavyrulewidth+\abovetopsep\relax}
		\begin{tabular}{S[table-format=1.2]S[table-format=3.0]}
\toprule
{Strom $I$ / \si{A}} & {Leistung $P$ / \si{nW}} \\
\midrule
                5.22 &                      161 \\
                5.01 &                      160 \\
                4.80 &                      158 \\
                4.60 &                      153 \\
                4.41 &                      145 \\
                4.20 &                      132 \\
                4.00 &                      126 \\
                3.80 &                      121 \\
                3.60 &                      113 \\
                3.40 &                      112 \\
                3.20 &                       97 \\
                3.00 &                       89 \\
                2.80 &                       74 \\
                2.60 &                       69 \\
\bottomrule
\end{tabular}

	\end{minipage} 
	\begin{minipage}[t]{0.3\textwidth}
		\centering
		\vspace*{-\dimexpr\baselineskip+\heavyrulewidth+\abovetopsep\relax}
		\begin{tabular}{S[table-format=1.2]S[table-format=3.0]}
\toprule
{Strom $I$ / \si{A}} & {Leistung $P$ / \si{nW}} \\
\midrule
                2.40 &                       60 \\
                2.20 &                       52 \\
                2.00 &                       44 \\
                1.80 &                       35 \\
                1.60 &                       33 \\
                1.40 &                       29 \\
                1.20 &                       22 \\
                1.00 &                       17 \\
                0.80 &                       11 \\
                0.60 &                        6 \\
                0.40 &                        3 \\
                0.20 &                        2 \\
                0.00 &                        0 \\
\bottomrule
\end{tabular}

	\end{minipage}
	\captionof{table}{Messwerte zur Bestimmung des Einflusses des Spulenstroms~$I$ mit einem Fehler von $\Delta I = \SI{0.01}{\ampere}$ auf die Fluoreszenzleistung der gefangenen Atome $P$ mit $\Delta P = \SI{5}{nW}$.}
	\label{tab:magnetic_field}
\end{figure}

Die Abbildung kann im wesentlichen in drei unterschiedliche Bereiche eingeteilt werden.
Zum einen ist dies der Bereich kleiner Ströme von \num{0} bis etwa \SI{1}{\ampere}, bei denen nur eine kleine Fluoreszenzleistung beobachtet werden kann, die nur langsam mit dem Spulenstrom ansteigt.
In diesem Strombereich ist der Potentialtopf zu klein um eine signifikante Anzahl von Atomen einzufangen.
Ab einem Strom von \SI{1}{\ampere} steigt die Fluoreszenzleistung stark mit dem Spulenstrom und die gefangenen Atome sind mithilfe einer Kamera klar zu erkennen.
An diesem Punkt können die Rubidiumatome effizient durch das Fallenpotential gefangen werden.
Übersteigt der Spulenstrom \SI{4.4}{\ampere} so stellt man fest, dass die Falle scheinbar in Sättigung übergeht.
\korr{Argument für die Sättigung}
Darüber hinaus finden sich einige Sprünge in den Messdaten deren Ursache bei dem Laser für das optische Pumpen der Rubidiumatome auf den Kühlübergang zu suchen sind.
Dieser kann aufgrund eines schlechten Fehlersignals nicht durch die Lockbox auf den Übergang festgestellt werden und somit leichte Abweichungen während der Messung verursachen.


\subsection{Verhalten des Ladevorgangs}
Folglich soll das Verhalten des Ladevorgangs der Atomfalle untersucht werden.
Nun werden die gefangenen Atome durch eine Linse auf eine Photodiode abgebildet, welche eine zur Fluoreszenz der gefangenen Atome proportionale Spannung erzeugt, die folglich mit einem Oszilloskop gemessen werden kann.
Um einen Ladevorgang der Falle zu starten wird zunächst der Spulenstrom durch das Anti-Helmholtz-Spulenpaar deaktiviert sodass keine MOT mehr zu beobachten ist.
Danach wird das Magnetfeld reaktiviert und auf dem Oszilloskop kann ein exponentieller Anstieg der Fluoreszenz der gefangenen Atome beobachtet werden.
Dieser Anstieg verhält sich wie die Aufladung eines Kondensators und folglich wird die Anpassungshypothese
\begin{align}
	U(t) =  U_0 \left[ 1 - \exp\left( - \frac{t - t_0}{\tau} \right) \right] \cdot \Theta(t - t_0) + \mathrm{BG}
	\label{eq:loading_fit}
\end{align}
mit dem maximalen Fluoreszenzsignal~$U_0$ der Photodiode, der Zeitkonstante der Aufladung~$\tau$, dem Beginn des Ladevorgangs~$t_0$, der Heaviside-Funktion~$\Theta$ und einem konstanten Untergrund~$\mathrm{BG}$ verwendet.
In Abbildung \ref{fig:loading} wurde beispielhaft eine Anpassung an die Ladekurve durchgeführt.
Insgesamt wurden 11 Ladekurven zur Bestimmung der mittleren Zeitkonstante~$\bar{\tau}$ vermessen und die resultierenden Anpassungsparameter wurden in Tabelle~\ref{tab:loading_params} zusammengefasst.
\begin{figure}[htbp]
	\centering
	\includegraphics{./figures/loading/loading11.pdf}
	\caption{Anpassung einer Ladekurve gemäß der Anpassungshypothese in Gleichung \eqref{eq:loading_fit} an den zeitlichen Verlauf des Photodiodensignal~$U$ während des Ladevorgangs der MOT.}
	\label{fig:loading}
	
	\vspace{1cm}
	
	\resizebox{\textwidth}{!}{
		\begin{tabular}{S[table-format=2.2]S[table-format=1.2]S[table-format=3.2]S[table-format=1.2]S[table-format=3.2]S[table-format=1.2]S[table-format=2.2]S[table-format=1.2]}
\toprule
{$U_0$ / \si{mV}} & {$\Delta U_0$ / \si{mV}} & {$\tau$ / \si{ms}} & {$\Delta \tau$ / \si{ms}} & {$t_0$ / \si{ms}} & {$\Delta t_0$ / \si{ms}} & {$\mathrm{BG}$ / \si{mV}} & {$\Delta \mathrm{BG}$ / \si{mV}} \\
\midrule
            52.29 &                     0.22 &             105.09 &                      0.97 &             48.99 &                     0.71 &                     26.82 &                             0.20 \\
            52.52 &                     0.16 &             110.03 &                      0.94 &             80.53 &                     0.63 &                     27.07 &                             0.15 \\
            57.32 &                     0.17 &              97.52 &                      0.96 &            133.15 &                     0.62 &                     36.64 &                             0.14 \\
            55.88 &                     0.20 &              95.77 &                      0.95 &             86.37 &                     0.65 &                     37.41 &                             0.18 \\
            55.92 &                     0.22 &              99.23 &                      1.04 &            100.50 &                     0.72 &                     37.79 &                             0.20 \\
            50.23 &                     0.17 &              98.55 &                      1.17 &            141.01 &                     0.75 &                     38.57 &                             0.15 \\
            45.85 &                     0.27 &             103.74 &                      1.34 &             45.97 &                     0.99 &                     38.30 &                             0.26 \\
            35.51 &                     0.19 &             100.33 &                      1.64 &            109.21 &                     1.08 &                     35.35 &                             0.17 \\
            59.11 &                     0.16 &              96.21 &                      0.80 &             95.11 &                     0.53 &                     39.52 &                             0.14 \\
            56.95 &                     0.24 &              95.02 &                      0.95 &            126.57 &                     0.69 &                     38.20 &                             0.23 \\
            60.40 &                     0.19 &              94.33 &                      0.82 &             78.27 &                     0.57 &                     38.40 &                             0.17 \\
\bottomrule
\end{tabular}

	}
	\captionof{table}{Ergebnisse der Anpassungen gemäß Gleichung \eqref{eq:loading_fit} an die vermessenen Ladekurven der MOT.}
	\label{tab:loading_params}
\end{figure}

Schließlich kann aus den 11 vermessenen Ladekurven die mittlere Zeitkonstante~$\bar{\tau}$ des Aufladeprozesses bestimmt werden.
Diese beträgt
\begin{align*}
	\bar{\tau} = \SI{99.6 +- 4.9}{ms} \, \text{,}
\end{align*}
wobei der Fehler aus der Stichprobenvarianz abgeschätzt wird.

\subsubsection{Bestimmung des Rb-Rb Wirkungsquerschnittes}
Anschließend kann die Messung der mittleren Zeitkonstante~$\bar{\tau}$ dazu genutzt werden den Wirkungsquerschnitt von Rb-Rb Kollisionen zu berechnen.
Dies ist möglich, da gemäß \cite{wieman} die Zeitkonstante des Aufladeprozesses ebenfalls der mittleren Zeit entspricht, die ein Rubidiumatom in der Falle gefangen bleibt.
Demnach kann die Zeitkonstante unter Vernachlässigung weiterer Spurengase durch 
\begin{align*}
	\frac{1}{\tau} = n_\mathrm{Rb}  \sigma_\mathrm{Rb}  v_\mathrm{Rb}
\end{align*}
beschrieben werden \cite{wieman}, wobei die Rubidiumdichte~$n_\mathrm{Rb}$, der Rb-Rb Wirkungsquerschnitt~$\sigma_\mathrm{Rb}$ und die mittlere Geschwindigkeit der Rubidiumatome~$v_\mathrm{Rb}$ gegeben ist.
Somit kann der Wirkungsquerschnitt durch
\begin{align}
	\sigma_\mathrm{Rb} = \frac{1}{\tau n_\mathrm{Rb} v_\mathrm{Rb}}
	\label{eq:cross_sec}
\end{align}
berechnet werden.
Dazu wird zunächst die Rubidiumdichte~$n_\mathrm{Rb}$ benötigt, welche aus dem Druck in der Vakuumkammer~$p$ und der Temperatur~$T$ des Gases mithilfe der idealen Gasgleichung berechnet werden kann.
Dabei wurde die Temperatur~$T$ zu $\SI{24.6 +- 0.5}{\degreeCelsius}$ mithilfe eines Thermometers bestimmt.
Der Druck~$p$ in der Vakuumkammer muss auf $\SI{1.1 +- 0.2e-7}{mbar}$ abgeschätzt werden, da ein Fehler an der Druckanzeige der Vakuumpumpe vorlag.
Dann kann gemäß der idealen Gasgleichung die Rubidiumdichte durch
\begin{align*}
	n_\mathrm{Rb} = \frac{p}{k_\mathrm{B} T}
\end{align*}
bestimmt werden.
Unter Verwendung des abgeschätzten Drucks und der Temperatur des Gases erhält man
\begin{align*}
	n_\mathrm{Rb} = \SI{2.68 +- 0.49e15}{\per\meter\cubed} \, \text{,}
\end{align*}
wobei der Fehler durch \textsc{Gauß}sche Fehlerfortpflanzung berechnet wurde.
Anschließend kann mithilfe der Maxwell-Boltzmann-Verteilung die mittlere Geschwindigkeit der Rubidiumatome~$v_\mathrm{Rb}$ durch
\begin{align*}
	v_\mathrm{Rb} = \sqrt{\frac{8 k_\mathrm{B} T}{\pi m_\mathrm{Rb}}} 
\end{align*}
berechnet werden.
Mit der gemessenen Temperatur des Gases und der Masse von \isotope[85]{Rb}~$m_\mathrm{Rb} = \SI{1.41e-25}{kg}$ \cite{handbook_spectroscopic_data} erhält man
\begin{align*}
	v_\mathrm{Rb} = \SI{272.5 +- 0.3}{\meter\per\second} \, \text{.}
\end{align*}
Schließlich können diese Werte und die im vorigen Abschnitt vermessene mittlere Zeitkonstante in Gleichung \eqref{eq:cross_sec} eingesetzt werden um
\begin{align*}
	\sigma_\mathrm{Rb} &= \SI{1.37 +- 0.26e-13}{\centi\meter\squared}\\
\end{align*}
zu erhalten.

Eine Abschätzung für den tatsächlichen Wert des Rb-Rb Wirkungsquerschnittes kann aus den Ergebnissen von \cite{force_in_mot} erfolgen.
Die Interaktion der Rubidiumatome erfolgt durch resonante Dipol-Dipol-Wechselwirkung und der Wirkungsquerschnitt liegt dann in der Größenordnung von etwa~$10^{-13} \, \si{\centi\meter\squared}$.
Der in diesem Versuch bestimmte Wert fällt ebenfalls in diesen Bereich, sollte jedoch nur als grobe Abschätzung angesehen werden, da der genaue Druck in der Vakuumkammer nicht bekannt ist.


\subsection{Verstimmung der Laserfrequenzen}
Letztlich soll die Fluoreszenz der magneto-optischen Falle in Abhängigkeit der Frequenz von Kühllaser sowie Pumplaser untersucht werden.
Dazu wird jeweils einer der Laser mithilfe der Lockbox auf eine bestimmte Frequenz festgelegt, während der andere im Scan-Modus mit einer Frequenz von \SI{20}{mHz} betrieben wird.
Dadurch wird die Frequenz des Lasers in Abhängigkeit von Scan-Offset und Amplitude über einen kleinen Bereich variiert und mithilfe einer Photodiode kann die Fluoreszenz der MOT gemessen werden.
Dabei ist es wichtig die Scan-Frequenz so klein zu wählen, dass die Periode des Scan-Signals wesentlich kleiner als die typische Ladezeit von etwa \SI{0.1}{s} der MOT.
Darüber hinaus wird eine Untergrundmessung mit deaktiviertem Magnetfeld durchgeführt, um die Fluoreszenz des heißen Hintergrundgases entfernen zu können.
Dieser Untergrund kann mithilfe des Spektrums aus der Rb-Spektroskopie an das Fluoreszenzsignal der MOT angepasst werden und folglich abgezogen werden.

\subsubsection{Variation der Kühllaserfrequenz}
\label{sec:detuning_cooling}
Zunächst wird die Frequenz des Pumplasers fixiert, indem die Scan-Amplitude minimiert wird und mithilfe des Scan-Offset der Pumpübergang auf Resonanz gestellt wird.
Im Gegensatz zum Kühllaser konnte der Pumplaser aufgrund eines stark Verrauschten Fehlersignals nicht mithilfe der Rückkopplungsschleife fixiert werden.
Anschließend wird der Scan-Bereich des Kühllasers auf die Umgebung des Kühlübergangs eingestellt und mithilfe des Oszilloskops kann sowohl die Rb-Spektroskopie sowie das Fluoreszenzsignal der Photodiode dargestellt werden.
Nun kann die Fluoreszenz der MOT in Abhängigkeit der Verstimmung des Kühllasers gemessen werden.
Dazu wird jeweils eine Messung mit und eine ohne Magnetfeld durchgeführt, um die Hintergrundfluoreszenz in der Analyse entfernen zu können.

In Abbildung~\ref{fig:detuning_cooling} wurde sowohl das Rb-Spektrum, als auch das Fluoreszenzsignal der MOT und dessen Untergrund dargestellt.
Weiterhin wurde das Signal nach Subtraktion des Untergrundes dargestellt.
An das Signal der Rb-Spektroskopie wurde folglich eine Kurve bestehend aus drei Lorentz-Kurven und einem polynomialen Untergrund 4.\ Ordnung angepasst.
Dies ermöglicht die Bestimmung der Frequenzen der drei Übergänge und wurde genutzt um das Untergrundsignal an das Fluoreszenzsignal der MOT anzupassen und folglich von diesem abzuziehen.
Darüber hinaus können die bekannten Frequenzen der Übergänge genutzt werden um eine Frequenzeichung durchzuführen.
Dazu nutzt man die Verstimmung $\delta_{1,2}$ der beiden Überkreuzungs-Signale (im Spektrum links) gegenüber dem Kühlübergang, welche in \cite{script} gegeben ist durch $\delta_1 = \SI{-92.0}{MHz}$ ($F=3 \rightarrow F^\prime=2,4$) und $\delta_2 = \SI{-60.3}{MHz}$ ($F=3 \rightarrow F^\prime=3,4$).
Schließlich wird an das Fluoreszenzsignal, welches bereits vom Untergrund bereinigt wurde, eine Gauß-Funktion angepasst, welche die Zentralfrequenz und somit die Rotverstimmung gegenüber dem Kühlübergang liefert.
\begin{figure}[h]
	\centering
	\includegraphics{./figures/detuning_cooling.pdf}
	\caption{Darstellung des Rubidium-Spektrums und der MOT-Fluoreszenz in Abhängigkeit der Verstimmung des Kühllasers. An das Spektrum wurde eine Funktion aus drei Lorentz-Kurven und einem polynomialen Untergrund 4.\ Ordnung angepasst und an das Fluoreszenzsignal der MOT eine Gauß-Kurve.}
	\label{fig:detuning_cooling}
\end{figure}
Im Folgenden soll auf eine Angabe sämtlicher Anpassungsparameter verzichtet werden und nur die Parameter erwähnt werden, die für die weitere Auswertung relevant sind.
Zum einen ist dies die Verstimmung für optimale MOT-Operation, welche aus dem Abstand des Mittelpunkts der Gauß-Kurve, welche an das Fluoreszenzsignal der MOT angepasst wurde, zu dem Zentrum der Lorentz-Kurve des Kühlübergangs.
Diese Verstimmung beträgt demnach
\begin{align*}
	\delta = \SI{-11.6 +- 0.4}{MHz} \, \text{,}
\end{align*}
wobei das negative Vorzeichen die erwartete Rotverschiebung signalisiert.
Diese Verstimmung wurde bereits in Abschnitt~\ref{sec:atomzahl} genutzt, um die Anzahl der gefangenen Atome in der Falle zu bestimmen.
Die für den Betrieb einer magneto-optischen Falle optimale Verstimmung~$\delta_\mathrm{opt.}$ ist gegeben durch
\begin{align*}
	\delta_\mathrm{opt.} = -\frac{\Gamma}{2} \, \text{,}
\end{align*}
wobei~$\Gamma$ die natürliche Linienbreite des Kühlübergangs beschreibt \cite{foot}.
Die Linienbreite kann aus der vollen Halbwertsbreite der Lorentz-Kurve an den jeweiligen Übergang gewonnen werden und man erhält für den Kühlübergang
\begin{align*}
	\Gamma = \SI{25.7 +- 1.0}{MHz}
\end{align*}
und folglich für die optimale Verstimmung
\begin{align*}
	\delta_\mathrm{opt.} = \SI{-12.8 +- 0.5}{MHz} \, \text{.}
\end{align*}






\section{Fazit}


\FloatBarrier
% BIBLIOGRAPHIE
\vspace{\fill}
% Maximale Anzahl der Einträge in Klammer
% Zitieren mit \cite{lamport94}
\begin{thebibliography}{19}
\bibitem{anleitung}
	\emph{Advanced Laboratory Course (physics601): Description of Experiments}, BONN-AT-2016-01MP, Universität Bonn, Januar 2016

\bibitem{wieman}
	C.\ Wieman, G.\ Flowers, S.\ Gilbert,
	\emph{Inexpensive laser cooling and trapping experiment for undergraduate laboratories},
	Am.\ J.\ Phys.\ \textbf{63} (4), April 1995.

\bibitem{handbook_spectroscopic_data}
	J.\ E.\ Sansonetti, W.\ C.\ Martin,
	\emph{Handbook of Basic Atomic Spectroscopic Data},
	National Institute of Standards and Technology (NIST), \url{http://physics.nist.gov/PhysRefData/Handbook/Tables/rubidiumtable1.htm} (Letzter Aufruf: 7.\ April 2016).

\bibitem{force_in_mot}
	A.\ M.\ Steane, M.\ Chowdhury, C.\ J.\ Foot,
	\emph{Radiation force in the magneto-optical trap},
	J.\ Opt.\ Soc.\ Am.\ B \textbf{9}, 2142 (1992).

\bibitem{script}
	Skript zum Versuch,
	\emph{FP Experiment: Rubidium MOT},
	(Stand: Januar 2014).

\bibitem{foot}
	C.\ J.\ Foot,
	\emph{Atomic Physics},
	Oxford University Press, 2005.
	
\bibitem{steck}
	D.\ A.\ Steck,
	\emph{Rubidium 85 D Line Data},
	\url{http://steck.us/alkalidata} (Revision 2.1.6, 20.\ September 2013)
\end{thebibliography}

% APPENDIX
\begin{appendix}
\newpage
\section{Anhang}
\end{appendix}

\end{document}
