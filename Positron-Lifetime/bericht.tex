% PAKETE UND DOKUMENTKONFIGURATION
\documentclass[11pt, a4paper]{article}

% Encoding für Umlaute
\usepackage[utf8]{inputenc}
\usepackage[T1]{fontenc}

% Silbentrennung
\usepackage[ngerman]{babel}

% erweiterte Matheumgebungen und Formelnummer mit Sectionnummer
\usepackage{amsmath}
\numberwithin{equation}{section}

% Braket Notation
\usepackage{braket}
\usepackage{isotope}
\usepackage[version=4]{mhchem}
\usepackage{tensor}
\usepackage{slashed}

% zusätzliche mathematische Schriftarten
\usepackage{amsfonts}

% verschiedene mathematische Symbole
\usepackage{amssymb}

% Einheiten setzen z.B. \SI{10}{\kilo\gram\meter\per\second\squared}
% Fehler: \SI{10 +- 0,2e-4}{\metre}
\usepackage{siunitx}
\sisetup{
  output-decimal-marker={,},
  separate-uncertainty
}

% Einheitendefinitionen
\DeclareSIUnit{\skt}{Skt.}
\DeclareSIUnit{\gauss}{G}
\DeclareSIUnit{\division}{div.}

% Operatordefinitionen
\DeclareMathOperator{\erf}{erf}

% Randbreiten
\usepackage[left=3.5cm,right=3.5cm,top=3cm,bottom=3cm,twoside]{geometry}

% Bilder einfügen
\usepackage{graphicx}
\usepackage[percent]{overpic}

% Textfarbe
\usepackage{color}

% Verweise innerhalb des Dokuments
\usepackage{hyperref}
\hypersetup{
	colorlinks = true,
	allcolors = {black}
}

% bessere Tabellenlayouts
\usepackage{booktabs}
\usepackage{multirow}
\usepackage{multicol}

% Seitenlayout (Kopfzeile)
\usepackage{fancyhdr}

% Float Barriers
\usepackage{placeins}

% Pakete für gedrehte Subfigures
\usepackage{caption}
\usepackage{subcaption}
\usepackage{rotating}

% Paket für textumflossene Abbildungen und Tabellen
\usepackage{wrapfig}

\usepackage{float}

% Caption-Setup
\captionsetup{font={small}}
\renewcommand{\thefigure}{\thesection.\arabic{figure}}
\renewcommand{\thesubfigure}{\alph{subfigure}}
\renewcommand{\thetable}{\thesection.\arabic{table}}
\renewcommand{\thesubtable}{\alph{subtable}}

% Manuelle Silbentrennung
\hyphenation{Par-ton-ver-teil-ungs-funk-tio-nen}

% Tiefe des Inhaltsverzeichnisses (Level: 1 sections, 2 subsections,
% 3 subsubsections)
\setcounter{tocdepth}{3}

% Code Einbettung
\usepackage{listings}
\definecolor{dkgreen}{rgb}{0,0.6,0}
\definecolor{gray}{rgb}{0.5,0.5,0.5}
\definecolor{mauve}{rgb}{0.58,0,0.82}

\lstset{frame=tb,
	aboveskip=3mm,
	belowskip=3mm,
	showstringspaces=false,
	columns=flexible,
	basicstyle={\ttfamily},
	numbers=left,
	numbersep=4pt,
	numberstyle=\tiny\color{black},
	keywordstyle=\color{blue},
	commentstyle=\color{dkgreen},
	stringstyle=\color{mauve},
	breaklines=true,
	breakatwhitespace=true,
	tabsize=3,
	xleftmargin=1em,
	xrightmargin=0.8em
}

% FANCYHDR SETUP
\pagestyle{fancy}
\fancyhead[EL,OR]{\thepage}
\fancyhead[ER]{\leftmark}
\fancyhead[OL]{\rightmark}
\setlength{\headheight}{13.6pt}

\renewcommand{\sectionmark}[1]{
\markboth{\thesection{} #1}{\thesection{} #1}
}
\renewcommand{\subsectionmark}[1]{
\markright{\thesubsection{} #1}
}

\newcommand{\korr}[1]{{\color{red}(#1)}}

% DOKUMENTINFORMATIONEN
\title{K225 \\ Lebensdauer von Positronen in Metallen und Isolatoren}

\author{Christopher Deutsch\footnote{christopher.deutsch@uni-bonn.de} \and Christian Bespin\footnote{christian.bespin@uni-bonn.de}}

\date{\today}

\begin{document}

\begin{titlepage}

\maketitle

% DURCHFÜHRUNGSDATUM UND TUTOR
\begin{center}
\begin{tabular}{l r}
Durchführung: & 22. März 2016 \\
Gruppe: & P8 \\
Tutor: & Martin Urban
\end{tabular}
\end{center}

% ZUSAMMENFASSUNG
\begin{abstract}
\noindent 
\end{abstract}

\end{titlepage}

% INHALTSVERZEICHNIS
\tableofcontents
% Neue Seite nach TOC
\newpage

% INHALT VERSUCHSPROTOKOLL
\section{Theorie}

\subsection{Einführung}

\subsection{Positronenquellen}

Anforderung: hohe Anzahl $e^+$ und hohe Energien, um in Probe einzudringen

Häufigste Quelle: \ce{^22Na} (\SI{90}{\percent} Positronenausbeute)

Zerfall in angeregtes \ce{^22Ne}, in Grundzustand über $\gamma$(\SI{1275}{keV}), was als Startsignal genutzt wird

Na Spektrum zeigen

\subsection{Positronen und Positronium}

\subsection{Gitterdefekte in Metallen}


\FloatBarrier
% BIBLIOGRAPHIE
\vspace{\fill}
% Maximale Anzahl der Einträge in Klammer
% Zitieren mit \cite{lamport94}
\begin{thebibliography}{19}
\bibitem{wiki_standardmodell}
	\emph{Wikimedia Commons, the free media repository}. \url{https://commons.wikimedia.org/wiki/File:Standard_Model_of_Elementary_Particles-de.svg} (Letzter Aufruf: 15. März 2016)
	
\bibitem{script}
	N. Möser, J. Meier, J.-W. Tsung, E. von Törne,
	\emph{Der ATLAS-Versuch: Eigenschaften von W-Bosonen und die Suche nach neuer Physik}.
\bibitem{pdg}
	K.A. Olive \textit{et al.} (Particle Data Group),
	\emph{The Review of Particle Physics},
	Chin. Phys. C, \textbf{38}, 090001 (2014).

\bibitem{electron_atlas}
	ATLAS Collaboration (Aad, Georges \textit{et al.}),
	\emph{Electron performance measurements with the ATLAS detector using the 2010 LHC proton-proton collision data},
	Eur.\ Phys.\ J.\ C72 (2012) 1909.

\bibitem{odr}
	SciPy,
	\emph{Orthogonal distance regression (\texttt{scipy.odr})},
	\url{http://docs.scipy.org/doc/scipy/reference/odr.html} (Letzter Aufruf: 14. März 2016).

\bibitem{error_prop}
	B. Ochoa, S. Belongie,
	\emph{Covariance Propagation for Guided Matching},
	\url{http://vision.cornell.edu/se3/wp-content/uploads/2014/09/ochoa06.pdf} (Letzter Aufruf: 14. März 2016).
	
\bibitem{cern}
	CERN, \emph{Computer generated image of the whole ATLAS detector (CERN-GE-0803012-06)}, \url{http://cds.cern.ch/record/1095924} (Letzter Aufruf: 15. März 2016)
\end{thebibliography}

% APPENDIX
\begin{appendix}
\newpage
\section{Anhang}


\end{appendix}

\end{document}
